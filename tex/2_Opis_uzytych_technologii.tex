\newpage
\section{Opis użytych technologii}		%2
%(W podpunktach dokonać krótkiej charakterystyki użytych technologii ) 


W niniejszym rozdziale przedstawiono szczegółową charakterystykę technologii wykorzystanych w ramach realizowanego projektu. Opis obejmuje funkcjonalność, zastosowania oraz znaczenie każdej z nich w kontekście budowy nowoczesnych środowisk IT.

\subsection{Active Directory Domain Services (AD DS)}
Active Directory Domain Services \footnote{Active Directory na Wikipedii\cite{ActiveDirectory}} (AD DS) to centralna usługa katalogowa firmy Microsoft, zaprojektowana w celu zarządzania zasobami sieciowymi w sposób efektywny i skalowalny. AD DS pozwala na tworzenie hierarchicznych struktur domenowych, które umożliwiają organizowanie użytkowników, komputerów, grup, drukarek i innych zasobów. Kluczowe funkcje AD DS obejmują:
\begin{itemize}
	\item Centralne uwierzytelnianie użytkowników i urządzeń,
    \item Zarządzanie politykami bezpieczeństwa poprzez Group Policy,
    \item Możliwość delegowania uprawnień administracyjnych,
    \item Integrację z innymi usługami sieciowymi, takimi jak DNS czy DHCP.
\end{itemize}
Dzięki AD DS organizacje mogą efektywnie zarządzać dużymi środowiskami IT, zwiększając bezpieczeństwo i wydajność operacyjną.

\subsection{Domain Name System (DNS)}
Domain Name System\footnote{Więcej informacji o DNS\cite{dns}} (DNS) to podstawowy element infrastruktury sieciowej, który odpowiada za zamianę nazw domenowych (np. www.przyklad.com) na adresy IP wykorzystywane przez urządzenia sieciowe. DNS umożliwia łatwe lokalizowanie serwerów, aplikacji i innych zasobów. Najważniejsze cechy DNS to:
\begin{itemize}
	\item Rozproszona struktura, która zapewnia skalowalność i niezawodność,
\item Obsługa zapytań o różne rekordy, takie jak A (adres IPv4), AAAA (adres IPv6), MX (serwer poczty), czy CNAME (aliasy),
\item Mechanizmy redundancji i buforowania, które zwiększają wydajność i odporność na awarie.
\end{itemize}
DNS jest fundamentem działania internetu i wielu aplikacji sieciowych, dlatego jego poprawna konfiguracja i zarządzanie mają kluczowe znaczenie.

\subsection{Dynamic Host Configuration Protocol (DHCP)}
Dynamic Host Configuration Protocol\footnote{Więcej informacji o DHCP\cite{dhcp}} (DHCP) automatyzuje proces przypisywania adresów IP oraz innych parametrów sieciowych urządzeniom w sieci. Dzięki DHCP administratorzy mogą eliminować ręczną konfigurację każdego urządzenia, co jest szczególnie istotne w dużych środowiskach. Funkcje DHCP obejmują:
\begin{itemize}
    \item Dynamiczne przydzielanie adresów IP z puli adresowej,
    \item Przekazywanie informacji o bramie domyślnej, serwerach DNS i czasie dzierżawy adresu,
    \item Obsługę rezerwacji adresów IP dla konkretnych urządzeń.
\end{itemize}
DHCP znacznie upraszcza zarządzanie sieciami, minimalizując ryzyko konfliktów adresów IP i błędów konfiguracyjnych.

\subsection{Group Policy Objects (GPO)}
Group Policy Objects\footnote{Strona poświęcona administracji GPO\cite{gpo}} (GPO) to mechanizm pozwalający administratorom na definiowanie i egzekwowanie ustawień konfiguracyjnych oraz polityk bezpieczeństwa w środowisku domenowym. Dzięki GPO możliwe jest centralne zarządzanie ustawieniami systemów operacyjnych, aplikacji i środowiska użytkownika. Przykładowe zastosowania GPO to:
\begin{itemize}
	\item Automatyczne wdrażanie oprogramowania na komputerach użytkowników,
    \item Konfiguracja polityk haseł i zabezpieczeń,
    \item Ustawienia pulpitu, mapowanie dysków sieciowych oraz drukarek.
\end{itemize}
GPO zapewniają elastyczność i kontrolę w dużych środowiskach, ułatwiając przestrzeganie standardów korporacyjnych.

\subsection{Windows Deployment Services (WDS)}
Windows Deployment Services \footnote{Poradnik do Windows Deployment Services\cite{wds}}(WDS) to narzędzie, które umożliwia administratorom instalację systemów operacyjnych na komputerach poprzez sieć, bez potrzeby używania nośników fizycznych. WDS wspiera scenariusze takie jak:
\begin{itemize}
	\item Wdrażanie systemów Windows w trybie bezobsługowym (unattended),
\item Obsługa obrazów systemowych w formacie WIM i VHD,
\item Konfiguracja obrazów rozruchowych i instalacyjnych.
\end{itemize}
WDS pozwala na efektywne wdrażanie systemów w środowiskach, w których konieczna jest szybka instalacja na wielu urządzeniach jednocześnie.

\subsection{RAID 1}
RAID 1\footnote{Więcej informacji na temat RAID 1 można znaleźć na Wikipedii\cite{raid}} to technologia macierzy dyskowej znana również jako mirroring (lustrzanie), która zapewnia pełną redundancję danych. Dane są zapisywane jednocześnie na dwóch dyskach, co gwarantuje ich dostępność w przypadku awarii jednego z nich. Kluczowe cechy RAID 1 obejmują:
\begin{itemize}
	\item Wysoką odporność na utratę danych dzięki pełnej kopii zapasowej,
\item Łatwość odtwarzania danych w przypadku awarii,
\item Lepszą wydajność odczytu dzięki możliwości równoczesnego odczytu z dwóch dysków.
\end{itemize}
RAID 1 jest idealnym rozwiązaniem dla aplikacji wymagających wysokiej niezawodności, takich jak serwery baz danych czy systemy krytyczne.

\subsection{iSCSI}
Internet Small Computer Systems Interface\footnote{Dodatkowe szczegóły dotyczące protokołu iSCSI są dostępne na Wikipedii.\cite{iscsi}} (iSCSI) umożliwia przesyłanie poleceń SCSI przez sieci IP, co pozwala na budowanie sieciowych systemów pamięci masowej (SAN). iSCSI zapewnia:
\begin{itemize}
	\item Niskie koszty wdrożenia w porównaniu z tradycyjnymi technologiami SAN,
\item Łatwość integracji z istniejącymi infrastrukturami sieciowymi,
\item Wysoką wydajność dzięki zastosowaniu dedykowanych protokołów.
\end{itemize}
iSCSI jest idealnym rozwiązaniem dla firm poszukujących skalowalnych i przystępnych cenowo technologii przechowywania danych.

\subsection{Serwer Wydruków (Print Server)}
Serwer Wydruków\footnote{Poradnik dotyczący serwera wydruków na stronie Microsoft'u\cite{print_server}} centralizuje zarządzanie drukarkami i zadaniami drukowania w środowisku sieciowym. Jego funkcje obejmują:
\begin{itemize}
	\item Udostępnianie drukarek w sieci lokalnej,
\item Monitorowanie zadań drukowania i zużycia materiałów eksploatacyjnych,
\item Zarządzanie uprawnieniami dostępu do drukarek.
\end{itemize}
Dzięki serwerowi wydruków administratorzy mogą zwiększyć wydajność i kontrolę nad procesami drukowania, jednocześnie obniżając koszty operacyjne.

\subsection{Serwer WWW}
Serwer WWW, taki jak Internet Information Services\footnote{Dokumentacja Internet Information Services\cite{iis}} (IIS), obsługuje aplikacje webowe oraz strony internetowe, umożliwiając ich udostępnianie użytkownikom w sieci lokalnej i internecie. Kluczowe funkcje serwera WWW to:
\begin{itemize}
	\item Obsługa protokołów HTTP, HTTPS, FTP i SMTP,
\item Integracja z technologiami ASP.NET i PHP,
\item Skalowalność i wsparcie dla aplikacji wymagających wysokiej wydajności.
\end{itemize}
Serwery WWW odgrywają istotną rolę w nowoczesnych infrastrukturach IT, umożliwiając realizację usług online.

\subsection{Failover Clustering z rolami SFS i DHCP Server}
Failover Clustering\footnote{Przegląd klastrowania awaryjnego\cite{failover_clustering}} to technologia zapewniająca wysoką dostępność aplikacji i usług poprzez połączenie serwerów w klastry. Role takie jak Scale-Out File Server(SFS) i DHCP Server umożliwiają:
\begin{itemize}
	\item Ciągłość działania usług w przypadku awarii jednego z węzłów klastra,
    \item Łatwą skalowalność w celu dostosowania do rosnących wymagań,
    \item Efektywne zarządzanie zasobami dyskowymi i sieciowymi.
\end{itemize}
Failover Clustering znajduje zastosowanie w krytycznych środowiskach, gdzie nieprzerwane działanie usług jest kluczowe.

\subsection{Automatyzacja - Skrypty PowerShell}
PowerShell\footnote{Więcej informacji o narzędziu PowerShell\cite{powershell}} to zaawansowane środowisko skryptowe i powłoka poleceń zaprojektowana z myślą o administracji systemami Windows. Funkcjonalności PowerShell obejmują:
\begin{itemize}
	\item Automatyzację zadań, takich jak zarządzanie użytkownikami, konfiguracja systemów i monitorowanie zasobów,
    \item Obsługę zdalnych operacji za pomocą protokołu WinRM,
    \item Tworzenie modułów i skryptów dostosowanych do specyficznych potrzeb organizacji.
\end{itemize}
PowerShell jest narzędziem nieodzownym dla administratorów IT, umożliwiającym realizację nawet najbardziej złożonych operacji w sposób efektywny i powtarzalny.
