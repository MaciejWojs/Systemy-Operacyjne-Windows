\newpage
\section{Opis użytych technologii}		%2
%(W podpunktach dokonać krótkiej charakterystyki użytych technologii ) 


W niniejszym rozdziale przedstawiono szczegółową charakterystykę technologii wykorzystanych w ramach realizowanego projektu. Opis obejmuje funkcjonalność, zastosowania oraz znaczenie każdej z nich w kontekście budowy nowoczesnych środowisk IT.

\subsection{Active Directory Domain Services (AD DS)}
Active Directory Domain Services \footnote{Active Directory na Wikipedii\cite{ActiveDirectory}} (AD DS) to centralna usługa katalogowa firmy Microsoft, zaprojektowana w celu zarządzania zasobami sieciowymi w sposób efektywny i skalowalny. AD DS pozwala na tworzenie hierarchicznych struktur domenowych, które umożliwiają organizowanie użytkowników, komputerów, grup, drukarek i innych zasobów. Kluczowe funkcje AD DS obejmują:
- Centralne uwierzytelnianie użytkowników i urządzeń,
- Zarządzanie politykami bezpieczeństwa poprzez Group Policy,
- Możliwość delegowania uprawnień administracyjnych,
- Integrację z innymi usługami sieciowymi, takimi jak DNS czy DHCP.
Dzięki AD DS organizacje mogą efektywnie zarządzać dużymi środowiskami IT, zwiększając bezpieczeństwo i wydajność operacyjną.

\subsection{Domain Name System (DNS)}
Domain Name System\footnote{Więcej informacji o DNS\cite{dns}} (DNS) to podstawowy element infrastruktury sieciowej, który odpowiada za zamianę nazw domenowych (np. www.przyklad.com) na adresy IP wykorzystywane przez urządzenia sieciowe. DNS umożliwia łatwe lokalizowanie serwerów, aplikacji i innych zasobów. Najważniejsze cechy DNS to:
- Rozproszona struktura, która zapewnia skalowalność i niezawodność,
- Obsługa zapytań o różne rekordy, takie jak A (adres IPv4), AAAA (adres IPv6), MX (serwer poczty), czy CNAME (aliasy),
- Mechanizmy redundancji i buforowania, które zwiększają wydajność i odporność na awarie.
DNS jest fundamentem działania internetu i wielu aplikacji sieciowych, dlatego jego poprawna konfiguracja i zarządzanie mają kluczowe znaczenie.

\subsection{Dynamic Host Configuration Protocol (DHCP)}
Dynamic Host Configuration Protocol\footnote{Więcej informacji o DHCP\cite{dhcp}} (DNS) (DHCP) automatyzuje proces przypisywania adresów IP oraz innych parametrów sieciowych urządzeniom w sieci. Dzięki DHCP administratorzy mogą eliminować ręczną konfigurację każdego urządzenia, co jest szczególnie istotne w dużych środowiskach. Funkcje DHCP obejmują:
- Dynamiczne przydzielanie adresów IP z puli adresowej,
- Przekazywanie informacji o bramie domyślnej, serwerach DNS i czasie dzierżawy adresu,
- Obsługę rezerwacji adresów IP dla konkretnych urządzeń.
DHCP znacznie upraszcza zarządzanie sieciami, minimalizując ryzyko konfliktów adresów IP i błędów konfiguracyjnych.

\subsection{Group Policy Objects (GPO)}
Group Policy Objects\footnote{Strona poświęcona administracji GPO\cite{gpo}} (DNS) (GPO) to mechanizm pozwalający administratorom na definiowanie i egzekwowanie ustawień konfiguracyjnych oraz polityk bezpieczeństwa w środowisku domenowym. Dzięki GPO możliwe jest centralne zarządzanie ustawieniami systemów operacyjnych, aplikacji i środowiska użytkownika. Przykładowe zastosowania GPO to:
- Automatyczne wdrażanie oprogramowania na komputerach użytkowników,
- Konfiguracja polityk haseł i zabezpieczeń,
- Ustawienia pulpitu, mapowanie dysków sieciowych oraz drukarek.
GPO zapewniają elastyczność i kontrolę w dużych środowiskach, ułatwiając przestrzeganie standardów korporacyjnych.

\subsection{Windows Deployment Services (WDS)}
Windows Deployment Services (WDS) to narzędzie, które umożliwia administratorom instalację systemów operacyjnych na komputerach poprzez sieć, bez potrzeby używania nośników fizycznych. WDS wspiera scenariusze takie jak:
- Wdrażanie systemów Windows w trybie bezobsługowym (unattended),
- Obsługa obrazów systemowych w formacie WIM i VHD,
- Konfiguracja obrazów rozruchowych i instalacyjnych.
WDS pozwala na efektywne wdrażanie systemów w środowiskach, w których konieczna jest szybka instalacja na wielu urządzeniach jednocześnie.

\subsection{RAID 1}
RAID 1 to technologia macierzy dyskowej znana również jako mirroring (lustrzanie), która zapewnia pełną redundancję danych. Dane są zapisywane jednocześnie na dwóch dyskach, co gwarantuje ich dostępność w przypadku awarii jednego z nich. Kluczowe cechy RAID 1 obejmują:
- Wysoką odporność na utratę danych dzięki pełnej kopii zapasowej,
- Łatwość odtwarzania danych w przypadku awarii,
- Lepszą wydajność odczytu dzięki możliwości równoczesnego odczytu z dwóch dysków.
RAID 1 jest idealnym rozwiązaniem dla aplikacji wymagających wysokiej niezawodności, takich jak serwery baz danych czy systemy krytyczne.

\subsection{iSCSI}
Internet Small Computer Systems Interface (iSCSI) umożliwia przesyłanie poleceń SCSI przez sieci IP, co pozwala na budowanie sieciowych systemów pamięci masowej (SAN). iSCSI zapewnia:
- Niskie koszty wdrożenia w porównaniu z tradycyjnymi technologiami SAN,
- Łatwość integracji z istniejącymi infrastrukturami sieciowymi,
- Wysoką wydajność dzięki zastosowaniu dedykowanych protokołów.
iSCSI jest idealnym rozwiązaniem dla firm poszukujących skalowalnych i przystępnych cenowo technologii przechowywania danych.

\subsection{Serwer Wydruków (Print Server)}
Serwer Wydruków centralizuje zarządzanie drukarkami i zadaniami drukowania w środowisku sieciowym. Jego funkcje obejmują:
- Udostępnianie drukarek w sieci lokalnej,
- Monitorowanie zadań drukowania i zużycia materiałów eksploatacyjnych,
- Zarządzanie uprawnieniami dostępu do drukarek.
Dzięki serwerowi wydruków administratorzy mogą zwiększyć wydajność i kontrolę nad procesami drukowania, jednocześnie obniżając koszty operacyjne.

\subsection{Serwer WWW}
Serwer WWW, taki jak Internet Information Services (IIS), obsługuje aplikacje webowe oraz strony internetowe, umożliwiając ich udostępnianie użytkownikom w sieci lokalnej i internecie. Kluczowe funkcje serwera WWW to:
- Obsługa protokołów HTTP, HTTPS, FTP i SMTP,
- Integracja z technologiami ASP.NET i PHP,
- Skalowalność i wsparcie dla aplikacji wymagających wysokiej wydajności.
Serwery WWW odgrywają istotną rolę w nowoczesnych infrastrukturach IT, umożliwiając realizację usług online.

\subsection{Failover Clustering z rolami SFS i DHCP Server}
Failover Clustering to technologia zapewniająca wysoką dostępność aplikacji i usług poprzez połączenie serwerów w klastry. Role takie jak Scale-Out File Server (SFS) i DHCP Server umożliwiają:
- Ciągłość działania usług w przypadku awarii jednego z węzłów klastra,
- Łatwą skalowalność w celu dostosowania do rosnących wymagań,
- Efektywne zarządzanie zasobami dyskowymi i sieciowymi.
Failover Clustering znajduje zastosowanie w krytycznych środowiskach, gdzie nieprzerwane działanie usług jest kluczowe.

\subsection{Automatyzacja - Skrypty PowerShell}
PowerShell to zaawansowane środowisko skryptowe i powłoka poleceń zaprojektowana z myślą o administracji systemami Windows. Funkcjonalności PowerShell obejmują:
- Automatyzację zadań, takich jak zarządzanie użytkownikami, konfiguracja systemów i monitorowanie zasobów,
- Obsługę zdalnych operacji za pomocą protokołu WinRM,
- Tworzenie modułów i skryptów dostosowanych do specyficznych potrzeb organizacji.
PowerShell jest narzędziem nieodzownym dla administratorów IT, umożliwiającym realizację nawet najbardziej złożonych operacji w sposób efektywny i powtarzalny.
