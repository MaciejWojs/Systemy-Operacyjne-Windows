	\newpage
\section{Procedury instalacyjne poszczególnych usług}		%4
% Procedury instalacyjne poszczególnych usług.
% (W podpunktach zamieścić polecenia dotyczące instalacji wdrażanych usług) 

\subsection{Instalacja serwera}

Projekt rozpoczęto od instalacji systemu Windows Server 2022 na serwerze zainstalowanym w maszynie wirtualnej. Proces instalacji serwera jest widoczny na \OznaczZdjecie[rysunku]{sdc-install}. Po zainstalowaniu systemu, nadano serwerowi nazwę \texttt{SDC06} oraz przydzielono adres IP 192.168.6.40/24 co jest widoczne na \OznaczZdjecie[zdjęciu]{sdc-adres}.

\fg{rys/SDC-instalacja/1.png}{Instalacja serwera SDC}{sdc-install}
\clearpage
\fg{rys/SDC-instalacja/2.png}{Adresacja serwera SDC}{sdc-adres}
\subsection{Instalacja usługi Active Directory Domain Services (AD DS)}
Dalszym krokiem była instalacja usługi Active Directory Domain Services (AD DS) na serwerze \texttt{SDC06}.

\subsubsection{Proces instalacji AD DS}
\begin{enumerate}
\item Otwarcie \texttt{Server Manager}
\item Wybranie \texttt{Manage} $\rightarrow$ \texttt{Add Roles and Features}
\item Kliknięcie \texttt{Next}
\item Wybranie \texttt{Role-based or feature-based installation}. \OznaczZdjecie{ad-ds1}
\item Wybranie serwera \texttt{SDC06} i kliknięcie \texttt{Next}. \OznaczZdjecie{ad-ds2}
\item Wybranie roli \texttt{Active Directory Domain Services}.\OznaczZdjecie{ad-ds3} i \OznaczZdjecie[rysunek]{ad-ds4}
\item Kliknięcie \texttt{Next} na ekranie z funkcjami. \OznaczZdjecie{ad-ds5}
\item Kliknięcie \texttt{Next} na ekranie z informacjami dotyczącymi AD. \OznaczZdjecie{ad-ds6}
\item Kliknięcie \texttt{Next} na ekranie podsumowania. \OznaczZdjecie{ad-ds7}
\item Kliknięcie \texttt{Install}
\item Ponowne uruchomienie serwera
\end{enumerate}

\fg{rys/uslugi/AD/instalacja/1.png}{Instalacja oparta na rolach lub funkcjach}{ad-ds1}
\fg{rys/uslugi/AD/instalacja/2.png}{Ekran z wyborem SDC}{ad-ds2}
\clearpage
\fg{rys/uslugi/AD/instalacja/3.png}{Dodanie usługi AD DS}{ad-ds3}
\fg{rys/uslugi/AD/instalacja/4.png}{Wybór roli serwera: AD DS}{ad-ds4}
\clearpage
\fg{rys/uslugi/AD/instalacja/5.png}{Wybór funcji serwera}{ad-ds5}
\fg{rys/uslugi/AD/instalacja/6.png}{Ekran z informacjami dotyczącymi AD}{ad-ds6}
\clearpage
\fg{rys/uslugi/AD/instalacja/7.png}{ekran podsumowania instalacji AD}{ad-ds7}
\fg{rys/uslugi/AD/instalacja/8.png}{Progress instalacji}{ad-ds8}
\clearpage
\fg{rys/uslugi/AD/instalacja/9.png}{Zainstalowane AD}{ad-ds9}


\subsubsection{Konfiguracja usługi AD DS}
Dalszym krokiem było utworzenie domeny \texttt{BinaryBuilders.ad} na serwerze oraz skonfigurowanie lasu w usłudze Active Directory Domain Services (AD DS). Proces ten obejmował kilka kluczowych etapów:
\begin{enumerate}
	\item Podniesienie poziomu serwera do kontrolera domeny. \OznaczZdjecie{ad-ds10}
	\item Stworzenie lasu. \OznaczZdjecie{ad-ds11}
	\item Wybór hasła dla trybu DSRM. \OznaczZdjecie{ad-ds12}
	\item Konfiguracja DNS. \OznaczZdjecie{ad-ds13}
	\item Dodanie nazwy NetBIOS \texttt{BINARYBUILDERS}. \OznaczZdjecie{ad-ds14}
	\item Wybór ścieżki do plików. \OznaczZdjecie{ad-ds15}
	\item Przegląd ustawień i wyświetlenie skryptu instalacji. \OznaczZdjecie{ad-ds16}
	\item Sprawdzenie wymagań komputera do instalacji AD. \OznaczZdjecie{ad-ds17}
	\item Poprawne sprawdzenie wymagań komputera do instalacji AD. \OznaczZdjecie{ad-ds18}
	\item Rozpoczęcie instalacji. \OznaczZdjecie{ad-ds19}
	\item Ekran informujący o wylogowaniu. \OznaczZdjecie{ad-ds20}

\end{enumerate}

\fg{rys/uslugi/AD/konfiguracja/1.png}{Podniesienie poziomu serwera do kontrolera domeny}{ad-ds10}
\fg{rys/uslugi/AD/konfiguracja/2.png}{Stworzenie lasu. Domena BinaryBuilders.ad}{ad-ds11}
\clearpage
\fg{rys/uslugi/AD/konfiguracja/3.png}{Wybór hasła dla trybu DSRM (domyślne opcje)}{ad-ds12}

\fg{rys/uslugi/AD/konfiguracja/4.png}{Konfiguracja DNS}{ad-ds13}
\clearpage
\fg{rys/uslugi/AD/konfiguracja/5.png}{Dodanie nazwy NetBIOS BINARYBUILDERS}{ad-ds14}
\fg{rys/uslugi/AD/konfiguracja/6.png}{Wybór ścieżki do plików (Domyślne opcje)}{ad-ds15}
\clearpage
\fg{rys/uslugi/AD/konfiguracja/7.png}{Przegląd ustawień i wyświetlenie skryptu instalacji}{ad-ds16}
\fg{rys/uslugi/AD/konfiguracja/8.png}{Sprawdzenie wymagań komputera do instalacji AD}{ad-ds17}
\clearpage
\fg{rys/uslugi/AD/konfiguracja/9.png}{Poprawne sprawdzenie wymagań}{ad-ds18}
\fg{rys/uslugi/AD/konfiguracja/10.png}{Rozpoczęcie instalacji}{ad-ds19}
\clearpage
\fg{rys/uslugi/AD/konfiguracja/11.png}{Ekran informujący o wylogowaniu w krótce}{ad-ds20}
\fg{rys/uslugi/AD/konfiguracja/12.png}{Zalogowanie się na konto administratora domeny}{ad-ds22}
\clearpage


\subsection{Instalacja usługi Domain Name System (DNS)}
Po zainstalowaniu usługi Active Directory Domain Services (AD DS) przystąpiono do instalacji usługi Domain Name System (DNS) na serwerze \texttt{SDC06}, ponieważ jest to niezbędne do prawidłowego funkcjonowania AD DS, a ja zapomniałem wykonać tego kroku wcześniej.
\subsubsection{Proces instalacji DNS}
\begin{enumerate}
	\item Otwarcie \texttt{Server Manager}
	\item Wybranie \texttt{Manage} $\rightarrow$ \texttt{Add Roles and Features}
	\item Kliknięcie \texttt{Next}
	\item Wybranie \texttt{Role-based or feature-based installation}. \OznaczZdjecie{dns1}
	\item Wybranie serwera \texttt{SDC06} i kliknięcie \texttt{Next}. \OznaczZdjecie{dns2}
	\item Wybranie roli \texttt{DNS Server}.\OznaczZdjecie{dns3} i \OznaczZdjecie[rysunek]{dns4}
	\item Kliknięcie \texttt{Next} na ekranie z funkcjami. \OznaczZdjecie{dns5}
	\item Kliknięcie \texttt{Next} na ekranie z informacjami dotyczącymi DNS.
	\item Kliknięcie \texttt{Next} na ekranie podsumowania. \OznaczZdjecie{dns6}
	\item Kliknięcie \texttt{Install} na ekranie z postępem instalacji. \OznaczZdjecie{dns7}
\end{enumerate}

\fg{rys/uslugi/DNS/instalacja/1.png}{Instalacja oparta na rolach lub funkcjach}{dns1}
\fg{rys/uslugi/DNS/instalacja/2.png}{Ekran z wyborem SDC}{dns2}
\clearpage
\fg{rys/uslugi/DNS/instalacja/3.png}{Dodanie usługi DNS}{dns3}
\fg{rys/uslugi/DNS/instalacja/4.png}{Wybór roli serwera: DNS}{dns4}
\clearpage

\fg{rys/uslugi/DNS/instalacja/5.png}{Wybór funcji serwera}{dns5}
\fg{rys/uslugi/DNS/instalacja/6.png}{Ekran podsumowania}{dns6}
\clearpage
\fg{rys/uslugi/DNS/instalacja/7.png}{Postęp instalacji DNS}{dns7}


\subsubsection{Proces konfiguracji DNS}
Po zainstalowaniu usługi DNS przystąpiono do konfiguracji usługi. Proces ten obejmował kilka kluczowych etapów:

\begin{enumerate}
	\item Otwarcie \texttt{DNS Manager} z poziomu \texttt{Server Manager} \OznaczZdjecie{dns8}
	\item Wybranie \texttt{SDC06.BinaryBuilders.ad} \OznaczZdjecie{dns9}
	\item Dodanie nowego hosta typu \texttt{A} o nazwie \texttt{www} i adresie IP serwera \texttt{SDC06}: 192.168.6.40 w strefie przeszukiwania do przodu. \OznaczZdjecie{dns10}
	\item Dodanie strefy przeszukiwania do tyłu. \OznaczZdjecie{dns11}
	\begin {enumerate}
	\item Kliknięcie \texttt{Next w oknie dodawania strefy przeszukiwania do tyłu} \OznaczZdjecie{dns12}
	\item Wybranie strefy podstawowej \texttt{BinaryBuilders.ad} \OznaczZdjecie{dns12}
	\item Wybranie strefy replikacji \texttt{To all DNS servers running on domain controllers in this domain: BinaryBuilders.ad} \OznaczZdjecie{dns13}
	\item Wybór IPv4 \OznaczZdjecie{dns14}
	\item Wybór nazwy lub adresu IP translacji \OznaczZdjecie{dns15}
	\item Wybór dynamicznego odświeżania strefy \OznaczZdjecie{dns16}
	\item Kliknięcie \texttt{Finish} \OznaczZdjecie{dns17}
	\end{enumerate}
	\item Dodanie wskaźnika ptr do strefy przeszukiwania do tyłu. \OznaczZdjecie{dns18}
\end{enumerate}

\fg{rys/uslugi/DNS/konfiguracja/1.png}{Otwarcie DNS Manager}{dns8}
\clearpage
\fg{rys/uslugi/DNS/konfiguracja/2.png}{Wybranie SDC06.BinaryBuilders.ad}{dns9}
\clearpage
\fg{rys/uslugi/DNS/konfiguracja/3.png}{Dodanie nowego hosta typu A}{dns10}
\clearpage
\fg{rys/uslugi/DNS/konfiguracja/4.png}{Dodanie strefy przeszukiwania do tyłu}{dns11}
\clearpage
\fg{rys/uslugi/DNS/konfiguracja/5.png}{Okno dodawania strefy}{dns12}
\clearpage
\fg{rys/uslugi/DNS/konfiguracja/6.png}{Wybór strefy podstawowej}{dns12}
\clearpage
\fg{rys/uslugi/DNS/konfiguracja/7.png}{Wybór strefy replikacji}{dns13}
\clearpage
\fg{rys/uslugi/DNS/konfiguracja/8.png}{Wybór IPv4}{dns14}
\clearpage
\fg{rys/uslugi/DNS/konfiguracja/9.png}{Wybór nazwy lub adresu IP translacji}{dns15}
\clearpage
\fg{rys/uslugi/DNS/konfiguracja/10.png}{Wybór dynamicznego odświeżania strefy}{dns16}
\clearpage
\fg{rys/uslugi/DNS/konfiguracja/11.png}{Zakończenie dodawania strefy}{dns17}
\clearpage
\fg{rys/uslugi/DNS/konfiguracja/12.png}{Dodanie wskaźnika ptr}{dns18}
\clearpage

\subsection{Skrypt dodający użytkowników i grupy do domeny}
Skrypt służy do automatycznego tworzenia użytkowników i grup w Active Directory na podstawie danych z pliku CSV. Skrypt jest napisany w języku PowerShell i wykorzystuje moduł Active Directory.
\subsubsection{Wytłumaczenie działania skryptu}

\paragraph{Główne komponenty}

\begin{verbatim}
$groupsOU = "OU=Grupy,$ou"
\end{verbatim}
Definiuje ścieżkę do kontenera grup w strukturze AD.

Skrypt iteruje przez każdy wiersz pliku CSV:
\begin{itemize}
    \item Tworzy nazwę użytkownika (imię.nazwisko)
    \item Normalizuje adres email
    \item Przypisuje stanowisko do grupy
\end{itemize}

\paragraph{Zarządzanie grupami}
\begin{itemize}
    \item Sprawdza istnienie grupy
    \item Tworzy nową grupę jeśli nie istnieje
    \item Zakres grupy: Global
    \item Lokalizacja: OU=Grupy
\end{itemize}

\paragraph{Zarządzanie użytkownikami}
Przy wykryciu istniejącej nazwy użytkownika:
\begin{itemize}
    \item Dodaje losowy numer (1-999)
    \item Aktualizuje adres email
    \item Sprawdza ponownie unikalność
\end{itemize}

Parametry nowego konta:
\begin{itemize}
    \item SamAccountName
    \item UserPrincipalName
    \item Pełna nazwa
    \item Email służbowy
    \item Dział
    \item Hasło (konwertowane na SecureString)
\end{itemize}

\paragraph{Ustawienia bezpieczeństwa}
\begin{itemize}
    \item Konto aktywne od razu
    \item Hasło nigdy nie wygasa
    \item Brak wymuszonej zmiany hasła
\end{itemize}


\paragraph{Wnioski}
Skrypt jest efektywnym narzędziem do automatyzacji procesu tworzenia kont użytkowników i grup w Active Directory. Może być łatwo dostosowany do różnych scenariuszy i wymagań.

\subsubsection{Skrypt}
\ListingFile{CSV}{CSV}
\clearpage
\subsubsection{Przykładowe dane do skryptu}
\ListingFile{Dane}{Dane}
\clearpage

\subsection{Instalacja usługi Dynamic Host Configuration Protocol (DHCP)}

\subsection{Instalacja usługi Group Policy Objects (GPO)}

\subsection{Instalacja usługi Windows Deployment Services (WDS)}

\subsection{Instalacja usługi RAID 1}

\subsection{Instalacja serwera WWW}

\subsection{Instalacja WordPress}

\subsection{Instalacja serwera DHCP}

