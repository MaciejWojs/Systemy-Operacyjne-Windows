	\newpage
\section{Procedury instalacyjne poszczególnych usług}		%4
% Procedury instalacyjne poszczególnych usług.
% (W podpunktach zamieścić polecenia dotyczące instalacji wdrażanych usług) 

\subsection{Instalacja serwera}

Projekt rozpoczęto od instalacji systemu Windows Server 2022 na serwerze zainstalowanym w maszynie wirtualnej. Proces instalacji serwera jest widoczny na \OznaczZdjecie[rysunku]{sdc-install}. Po zainstalowaniu systemu, nadano serwerowi nazwę \texttt{SDC06} oraz przydzielono adres IP 192.168.6.40/24 co jest widoczne na \OznaczZdjecie[zdjęciu]{sdc-adres}.

\fg{rys/SDC-instalacja/1.png}{Instalacja serwera SDC}{sdc-install}
\clearpage
\fg{rys/SDC-instalacja/2.png}{Adresacja serwera SDC}{sdc-adres}
\subsection{Instalacja usługi Active Directory Domain Services (AD DS)}
Dalszym krokiem była instalacja usługi Active Directory Domain Services (AD DS) na serwerze \texttt{SDC06}.

\subsubsection{Proces instalacji AD DS}
\begin{enumerate}
\item Otwarcie \texttt{Server Manager}
\item Wybranie \texttt{Manage} $\rightarrow$ \texttt{Add Roles and Features}
\item Kliknięcie \texttt{Next}
\item Wybranie \texttt{Role-based or feature-based installation}. \OznaczZdjecie{ad-ds1}
\item Wybranie serwera \texttt{SDC06} i kliknięcie \texttt{Next}. \OznaczZdjecie{ad-ds2}
\item Wybranie roli \texttt{Active Directory Domain Services}.\OznaczZdjecie{ad-ds3} i \OznaczZdjecie[rysunek]{ad-ds4}
\item Kliknięcie \texttt{Next} na ekranie z funkcjami. \OznaczZdjecie{ad-ds5}
\item Kliknięcie \texttt{Next} na ekranie z informacjami dotyczącymi AD. \OznaczZdjecie{ad-ds6}
\item Kliknięcie \texttt{Next} na ekranie podsumowania. \OznaczZdjecie{ad-ds7}
\item Kliknięcie \texttt{Install}
\item Ponowne uruchomienie serwera
\end{enumerate}

\fg{rys/uslugi/AD/instalacja/1.png}{Instalacja oparta na rolach lub funkcjach}{ad-ds1}
\fg{rys/uslugi/AD/instalacja/2.png}{Ekran z wyborem SDC}{ad-ds2}
\clearpage
\fg{rys/uslugi/AD/instalacja/3.png}{Dodanie usługi AD DS}{ad-ds3}
\fg{rys/uslugi/AD/instalacja/4.png}{Wybór roli serwera: AD DS}{ad-ds4}
\clearpage
\fg{rys/uslugi/AD/instalacja/5.png}{Wybór funcji serwera}{ad-ds5}
\fg{rys/uslugi/AD/instalacja/6.png}{Ekran z informacjami dotyczącymi AD}{ad-ds6}
\clearpage
\fg{rys/uslugi/AD/instalacja/7.png}{ekran podsumowania instalacji AD}{ad-ds7}
\fg{rys/uslugi/AD/instalacja/8.png}{Progress instalacji}{ad-ds8}
\clearpage
\fg{rys/uslugi/AD/instalacja/9.png}{Zainstalowane AD}{ad-ds9}


\subsubsection{Konfiguracja usługi AD DS}
Dalszym krokiem było utworzenie domeny \texttt{BinaryBuilders.ad} na serwerze oraz skonfigurowanie lasu w usłudze Active Directory Domain Services (AD DS). Proces ten obejmował kilka kluczowych etapów:
\begin{enumerate}
	\item Podniesienie poziomu serwera do kontrolera domeny. \OznaczZdjecie{ad-ds10}
	\item Stworzenie lasu. \OznaczZdjecie{ad-ds11}
	\item Wybór hasła dla trybu DSRM. \OznaczZdjecie{ad-ds12}
	\item Konfiguracja DNS. \OznaczZdjecie{ad-ds13}
	\item Dodanie nazwy NetBIOS \texttt{BINARYBUILDERS}. \OznaczZdjecie{ad-ds14}
	\item Wybór ścieżki do plików. \OznaczZdjecie{ad-ds15}
	\item Przegląd ustawień i wyświetlenie skryptu instalacji. \OznaczZdjecie{ad-ds16}
	\item Sprawdzenie wymagań komputera do instalacji AD. \OznaczZdjecie{ad-ds17}
	\item Poprawne sprawdzenie wymagań komputera do instalacji AD. \OznaczZdjecie{ad-ds18}
	\item Rozpoczęcie instalacji. \OznaczZdjecie{ad-ds19}
	\item Ekran informujący o wylogowaniu. \OznaczZdjecie{ad-ds20}

\end{enumerate}

\fg{rys/uslugi/AD/konfiguracja/1.png}{Podniesienie poziomu serwera do kontrolera domeny}{ad-ds10}
\fg{rys/uslugi/AD/konfiguracja/2.png}{Stworzenie lasu. Domena BinaryBuilders.ad}{ad-ds11}
\clearpage
\fg{rys/uslugi/AD/konfiguracja/3.png}{Wybór hasła dla trybu DSRM (domyślne opcje)}{ad-ds12}

\fg{rys/uslugi/AD/konfiguracja/4.png}{Konfiguracja DNS}{ad-ds13}
\clearpage
\fg{rys/uslugi/AD/konfiguracja/5.png}{Dodanie nazwy NetBIOS BINARYBUILDERS}{ad-ds14}
\fg{rys/uslugi/AD/konfiguracja/6.png}{Wybór ścieżki do plików (Domyślne opcje)}{ad-ds15}
\clearpage
\fg{rys/uslugi/AD/konfiguracja/7.png}{Przegląd ustawień i wyświetlenie skryptu instalacji}{ad-ds16}
\fg{rys/uslugi/AD/konfiguracja/8.png}{Sprawdzenie wymagań komputera do instalacji AD}{ad-ds17}
\clearpage
\fg{rys/uslugi/AD/konfiguracja/9.png}{Poprawne sprawdzenie wymagań}{ad-ds18}
\fg{rys/uslugi/AD/konfiguracja/10.png}{Rozpoczęcie instalacji}{ad-ds19}
\clearpage
\fg{rys/uslugi/AD/konfiguracja/11.png}{Ekran informujący o wylogowaniu w krótce}{ad-ds20}
\fg{rys/uslugi/AD/konfiguracja/12.png}{Zalogowanie się na konto administratora domeny}{ad-ds22}
\clearpage


\subsection{Instalacja usługi Domain Name System (DNS)}
Po zainstalowaniu usługi Active Directory Domain Services (AD DS) przystąpiono do instalacji usługi Domain Name System (DNS) na serwerze \texttt{SDC06}, ponieważ jest to niezbędne do prawidłowego funkcjonowania AD DS, a ja zapomniałem wykonać tego kroku wcześniej.
\subsubsection{Proces instalacji DNS}
\begin{enumerate}
	\item Otwarcie \texttt{Server Manager}
	\item Wybranie \texttt{Manage} $\rightarrow$ \texttt{Add Roles and Features}
	\item Kliknięcie \texttt{Next}
	\item Wybranie \texttt{Role-based or feature-based installation}. \OznaczZdjecie{dns1}
	\item Wybranie serwera \texttt{SDC06} i kliknięcie \texttt{Next}. \OznaczZdjecie{dns2}
	\item Wybranie roli \texttt{DNS Server}.\OznaczZdjecie{dns3} i \OznaczZdjecie[rysunek]{dns4}
	\item Kliknięcie \texttt{Next} na ekranie z funkcjami. \OznaczZdjecie{dns5}
	\item Kliknięcie \texttt{Next} na ekranie z informacjami dotyczącymi DNS.
	\item Kliknięcie \texttt{Next} na ekranie podsumowania. \OznaczZdjecie{dns6}
	\item Kliknięcie \texttt{Install} na ekranie z postępem instalacji. \OznaczZdjecie{dns7}
\end{enumerate}

\fg{rys/uslugi/DNS/instalacja/1.png}{Instalacja oparta na rolach lub funkcjach}{dns1}
\fg{rys/uslugi/DNS/instalacja/2.png}{Ekran z wyborem SDC}{dns2}
\clearpage
\fg{rys/uslugi/DNS/instalacja/3.png}{Dodanie usługi DNS}{dns3}
\fg{rys/uslugi/DNS/instalacja/4.png}{Wybór roli serwera: DNS}{dns4}
\clearpage

\fg{rys/uslugi/DNS/instalacja/5.png}{Wybór funcji serwera}{dns5}
\fg{rys/uslugi/DNS/instalacja/6.png}{Ekran podsumowania}{dns6}
\clearpage
\fg{rys/uslugi/DNS/instalacja/7.png}{Postęp instalacji DNS}{dns7}


\subsubsection{Proces konfiguracji DNS}
Po zainstalowaniu usługi DNS przystąpiono do konfiguracji usługi. Proces ten obejmował kilka kluczowych etapów:

\begin{enumerate}
	\item Otwarcie \texttt{DNS Manager} z poziomu \texttt{Server Manager} \OznaczZdjecie{dns8}
	\item Wybranie \texttt{SDC06.BinaryBuilders.ad} \OznaczZdjecie{dns9}
	\item Dodanie nowego hosta typu \texttt{A} o nazwie \texttt{www} i adresie IP serwera \texttt{SDC06}: 192.168.6.40 w strefie przeszukiwania do przodu. \OznaczZdjecie{dns10}
	\item Dodanie strefy przeszukiwania do tyłu. \OznaczZdjecie{dns11}
	\begin {enumerate}
	\item Kliknięcie \texttt{Next w oknie dodawania strefy przeszukiwania do tyłu} \OznaczZdjecie{dns12}
	\item Wybranie strefy podstawowej \texttt{BinaryBuilders.ad} \OznaczZdjecie{dns12}
	\item Wybranie strefy replikacji \texttt{To all DNS servers running on domain controllers in this domain: BinaryBuilders.ad} \OznaczZdjecie{dns13}
	\item Wybór IPv4 \OznaczZdjecie{dns14}
	\item Wybór nazwy lub adresu IP translacji \OznaczZdjecie{dns15}
	\item Wybór dynamicznego odświeżania strefy \OznaczZdjecie{dns16}
	\item Kliknięcie \texttt{Finish} \OznaczZdjecie{dns17}
	\end{enumerate}
	\item Dodanie wskaźnika ptr do strefy przeszukiwania do tyłu. \OznaczZdjecie{dns18}
\end{enumerate}

\fg{rys/uslugi/DNS/konfiguracja/1.png}{Otwarcie DNS Manager}{dns8}
\clearpage
\fg{rys/uslugi/DNS/konfiguracja/2.png}{Wybranie SDC06.BinaryBuilders.ad}{dns9}
\clearpage
\fg{rys/uslugi/DNS/konfiguracja/3.png}{Dodanie nowego hosta typu A}{dns10}
\clearpage
\fg{rys/uslugi/DNS/konfiguracja/4.png}{Dodanie strefy przeszukiwania do tyłu}{dns11}
\clearpage
\fg{rys/uslugi/DNS/konfiguracja/5.png}{Okno dodawania strefy}{dns12}
\clearpage
\fg{rys/uslugi/DNS/konfiguracja/6.png}{Wybór strefy podstawowej}{dns12}
\clearpage
\fg{rys/uslugi/DNS/konfiguracja/7.png}{Wybór strefy replikacji}{dns13}
\clearpage
\fg{rys/uslugi/DNS/konfiguracja/8.png}{Wybór IPv4}{dns14}
\clearpage
\fg{rys/uslugi/DNS/konfiguracja/9.png}{Wybór nazwy lub adresu IP translacji}{dns15}
\clearpage
\fg{rys/uslugi/DNS/konfiguracja/10.png}{Wybór dynamicznego odświeżania strefy}{dns16}
\clearpage
\fg{rys/uslugi/DNS/konfiguracja/11.png}{Zakończenie dodawania strefy}{dns17}
\clearpage
\fg{rys/uslugi/DNS/konfiguracja/12.png}{Dodanie wskaźnika ptr}{dns18}
\clearpage

\subsection{Skrypt dodający użytkowników i grupy do domeny}
Skrypt służy do automatycznego tworzenia użytkowników i grup w Active Directory na podstawie danych z pliku CSV. Skrypt jest napisany w języku PowerShell i wykorzystuje moduł Active Directory.
\subsubsection{Wytłumaczenie działania skryptu}

\paragraph{Główne komponenty}

\begin{verbatim}
$groupsOU = "OU=Grupy,$ou"
\end{verbatim}
Definiuje ścieżkę do kontenera grup w strukturze AD.

Skrypt iteruje przez każdy wiersz pliku CSV:
\begin{itemize}
    \item Tworzy nazwę użytkownika (imię.nazwisko)
    \item Normalizuje adres email
    \item Przypisuje stanowisko do grupy
\end{itemize}

\paragraph{Zarządzanie grupami}
\begin{itemize}
    \item Sprawdza istnienie grupy
    \item Tworzy nową grupę jeśli nie istnieje
    \item Zakres grupy: Global
    \item Lokalizacja: OU=Grupy
\end{itemize}

\paragraph{Zarządzanie użytkownikami}
Przy wykryciu istniejącej nazwy użytkownika:
\begin{itemize}
    \item Dodaje losowy numer (1–999)
    \item Aktualizuje adres email
    \item Sprawdza ponownie unikalność
\end{itemize}

Parametry nowego konta:
\begin{itemize}
    \item SamAccountName
    \item UserPrincipalName
    \item Pełna nazwa
    \item Email służbowy
    \item Dział
    \item Hasło (konwertowane na SecureString)
\end{itemize}

\paragraph{Ustawienia bezpieczeństwa}
\begin{itemize}
    \item Konto aktywne od razu
    \item Hasło nigdy nie wygasa
    \item Brak wymuszonej zmiany hasła
\end{itemize}


\paragraph{Wnioski}
Skrypt jest efektywnym narzędziem do automatyzacji procesu tworzenia kont użytkowników i grup w Active Directory. Może być łatwo dostosowany do różnych scenariuszy i wymagań.

\subsubsection{Skrypt}
\ListingFile{CSV}{CSV}
\clearpage
\subsubsection{Przykładowe dane do skryptu}
\ListingFile{Dane}{Dane}
\clearpage

\subsection{Instalacja usługi RAID 1}
Następnie przystąpiono do instalacji usługi RAID 1 na serwerze \texttt{SDC06}. Aby to zrobić, należy wykonać następujące kroki:
\begin{enumerate}
	\item Dodanie dwóch dysków o poejmnośći 30GB do maszyny wirtualnej w programie VirtualBox\footnote{Strona projektu VirtualBox\cite{VirtualBox}}. \OznaczZdjecie{raid1-1} %chktex 8
	\item Otwarcie \texttt{Zarządzania dyskami} i inicjalizacja dysków \OznaczZdjecie{raid1-2} %chktex 8
	\item Wynik inicjalizacji dysków. \OznaczZdjecie{raid1-3} %chktex 8
	\item Wybranie \texttt{New Mirrored Volume} jako typu RAID. \OznaczZdjecie{raid1-4} %chktex 8
	\item Dodanie drugiego dysku do stworzenia macierzy. \OznaczZdjecie{raid1-5} %chktex 8
	\item Wynik dodania drugiego dysku. \OznaczZdjecie{raid1-6} %chktex 8
	\item Brak przypisanej litery do macierzy RAID 1. \OznaczZdjecie{raid1-7} %chktex 8
	\item Wybranie braku formatowania. \OznaczZdjecie{raid1-8} %chktex 8
	\item Ekran potwierdzenia. \OznaczZdjecie{raid1-9} %chktex 8
	\item Wynik stworzenia macierzy. \OznaczZdjecie{raid1-10} %chktex 8
	\item Okno dodania litery dla RAID 1. \OznaczZdjecie{raid1-11} %chktex 8
	\item Ustawienie litery na \texttt{R:}. \OznaczZdjecie{raid1-12} %chktex 8
	\item Zakończenie konfigurowania macierzy – wynik. \OznaczZdjecie{raid1-13} %chktex 8
	\item Nadanie etykiety RAID 1. \OznaczZdjecie{raid1-14} %chktex 8
\end{enumerate}

\fg{rys/uslugi/RAID1/1.png}{Dodanie dysku do maszyny wirtualnej}{raid1-1} %chktex 8
\fg{rys/uslugi/RAID1/2.png}{Otwarcie Zarządzania dyskami i inicjalizacja dysków}{raid1-2} %chktex 8
\clearpage
\fg{rys/uslugi/RAID1/3.png}{Wynik inicjalizacji dysków}{raid1-3} %chktex 8
\fg{rys/uslugi/RAID1/4.png}{Wybranie New Mirrored Volume}{raid1-4} %chktex 8
\clearpage
\fg{rys/uslugi/RAID1/5.png}{Dodanie drugiego dysku w konfiguratorze}{raid1-5} %chktex 8
\fg{rys/uslugi/RAID1/6.png}{Wynik dodania drugiego dysku w konfuguratorze}{raid1-6} %chktex 8
\clearpage
\fg{rys/uslugi/RAID1/7.png}{Brak przypisania litery do macierzy Raid1}{raid1-7} %chktex 8
\fg{rys/uslugi/RAID1/8.png}{Wybranie braku formatowania}{raid1-8} %chktex 8
\clearpage
\fg{rys/uslugi/RAID1/9.png}{Ekran potwierdzenia}{raid1-9} %chktex 8
\fg{rys/uslugi/RAID1/10.png}{Wynik stworzenia macierzy}{raid1-10} %chktex 8
\clearpage
\fg{rys/uslugi/RAID1/11.png}{Okno dodania litery dla raid1}{raid1-11} %chktex 8
\clearpage
\fg{rys/uslugi/RAID1/12.png}{Ustawienie litery na R:}{raid1-12} %chktex 8
\fg{rys/uslugi/RAID1/13.png}{Wynik ustawienia litery}{raid1-13} %chktex 8
\clearpage
\fg{rys/uslugi/RAID1/14.png}{Nadanie etykiety RAID1}{raid1-14} %chktex 8
\clearpage

\subsection{Instalacja usługi iSCSI target server}
Instalacja usługi iSCSI umożliwia konfigurację urządzeń pamięci masowej przez sieć, co jest szczególnie przydatne w środowiskach wirtualnych i serwerowych. Poniżej przedstawiono kroki instalacji usługi iSCSI na serwerze \texttt{SDC06}:
\begin{enumerate}
	\item Otwarcie \texttt{Server Manager}
	\item Wybranie \texttt{Manage} $\rightarrow$ \texttt{Add Roles and Features}
	\item Kliknięcie \texttt{Next}
	\item Wybranie \texttt{Role-based or feature-based installation}. \OznaczZdjecie{iscsi1}
	\item Wybranie serwera \texttt{SDC06} i kliknięcie \texttt{Next}. \OznaczZdjecie{iscsi2}
	\item Wybranie roli \texttt{iSCSI Target Server}.\OznaczZdjecie{iscsi3} i \OznaczZdjecie[rysunek]{iscsi4}
	\item Kliknięcie \texttt{Next} na ekranie z funkcjami. \OznaczZdjecie{iscsi5}
	\item Kliknięcie \texttt{Next} na ekranie z informacjami dotyczącymi iSCSI.
	\item Kliknięcie \texttt{Next} na ekranie podsumowania. \OznaczZdjecie{iscsi6}
	\item Kliknięcie \texttt{Install} na ekranie z postępem instalacji. \OznaczZdjecie{iscsi7}
	\item Otwarcie \texttt{iSCSI Target Server} z poziomu \texttt{Server Manager} \OznaczZdjecie{iscsi8}
	\item Kliknięcie \texttt{New iSCSI Virtual Disk} \OznaczZdjecie{iscsi9}
	\item Wybranie dysku do konfiguracji $\rightarrow$ \texttt{R:} \OznaczZdjecie{iscsi10}
	\item Wybranie nazwy dysku $\rightarrow$ \texttt{Dysk1} \OznaczZdjecie{iscsi11}
	\item Wybranie rozmiaru dysku $\rightarrow$ 30GB \OznaczZdjecie{iscsi12}
	\item Wybranie nazwy celu $\rightarrow$ \texttt{dysk1-target} \OznaczZdjecie{iscsi13}
	\item Wybranie serwera dostępu $\rightarrow$ \texttt{stacja kliencka} \OznaczZdjecie{iscsi14}
	\item Kliknięcie \texttt{Next} na ekranie z funkcjami. \OznaczZdjecie{iscsi15}
	\item Kliknięcie \texttt{Next} na ekranie z informacjami dotyczącymi iSCSI.
	\item 
\end{enumerate}

\fg{rys/uslugi/iSCSI/1.png}{Instalacja oparta na rolach lub funkcjach}{iscsi1}
\fg{rys/uslugi/iSCSI/2.png}{Ekran z wyborem SDC}{iscsi2}
\clearpage
\fg{rys/uslugi/iSCSI/3.png}{Dodanie usługi iSCSI}{iscsi3}
\fg{rys/uslugi/iSCSI/4.png}{Ekran podsumowania}{iscsi4}
\clearpage
\fg{rys/uslugi/iSCSI/5.png}{Otworzenie New iSCSI Virtual Disk}{iscsi5}
\fg{rys/uslugi/iSCSI/6.png}{Wybór dysku do konfiguracji $\rightarrow$ R:}{iscsi6}
\clearpage
\fg{rys/uslugi/iSCSI/7.png}{Wybór nazwy dysku $\rightarrow$ Dysk1}{iscsi7}
\fg{rys/uslugi/iSCSI/8.png}{Wybór rozmiaru dysku $\rightarrow$ 30GB}{iscsi8}
\clearpage
\fg{rys/uslugi/iSCSI/9.png}{Wybór nazwy celu $\rightarrow$ dysk1-target}{iscsi9}
W tym etapie wystąpił błąd, a następnie w dalszej części projektu dysk oraz cel zostały usunięte, ponieważ nie były częścią pierwotnych założeń projektu.
\fg{rys/uslugi/iSCSI/10.png}{Wybór serwera dostępu $\rightarrow$ stacja kliencka}{iscsi10}
\clearpage
\fg{rys/uslugi/iSCSI/11.png}{Ekran potwierdzenia ustawień iSCSI}{iscsi11}
\fg{rys/uslugi/iSCSI/12.png}{Wynik konfiguracji iSCSI}{iscsi12}
\clearpage
\fg{rys/uslugi/iSCSI/13.png}{Kliknięcie na właściwości celu z poziomu Server Manager}{iscsi13}
\fg{rys/uslugi/iSCSI/14.png}{Brak stacji klieckiej}{iscsi14}
\clearpage
\fg{rys/uslugi/iSCSI/15.png}{Otworzenie iSCSI Inicjator na stacji klieckiej}{iscsi15}
\fg{rys/uslugi/iSCSI/16.png}{Dodanie adresu ip serwera SDC06 w zakładce Discover Portal}{iscsi16}
\clearpage
\fg{rys/uslugi/iSCSI/17.png}{Powrót na SDC06 w celu dodania stacji klienckiej jako inicjatora}{iscsi17}
\fg{rys/uslugi/iSCSI/18.png}{Powrót na komputer użytkownika w celu odświeżania wykrywanych urządzeń}{iscsi18}
\clearpage
\fg{rys/uslugi/iSCSI/19.png}{Połączenie z wykrytym dyskiem}{iscsi19}
\fg{rys/uslugi/iSCSI/20.png}{Automatyczna konfiguracja dysku w zakładce Woluminy i urządzenia}{iscsi20}
\clearpage
\fg{rys/uslugi/iSCSI/21.png}{Inicjalizacja dysku}{iscsi21}
\fg{rys/uslugi/iSCSI/22.png}{Wybór typu partycji}{iscsi22}
\clearpage
\fg{rys/uslugi/iSCSI/23.png}{Wynik działania dysku}{iscsi23}

\subsection{Instalacja klastra failover}
Na serwerze \texttt{SDC06} za pomocą narzędzia \texttt{Server Manager} zainstalowano rolę Failover Clustering.

\fg{rys/Klaster/1.png}{Usunięcie dysku iSCSI dysk1}{Klaster1}
\fg{rys/Klaster/2.png}{Wybór dysku do konfiguracji $\rightarrow$ R: w New iSCSI Virtual Disk}{Klaster2}
\clearpage
\fg{rys/Klaster/3.png}{Wybór nazwy dysku $\rightarrow$ klaster}{Klaster3}
\fg{rys/Klaster/4.png}{Wybór rozmiaru dysku $\rightarrow$ 30GB}{Klaster4}
\clearpage
\fg{rys/Klaster/5.png}{Nowy cel}{Klaster5}
\fg{rys/Klaster/6.png}{Nazwa celu klaster-SN1-SN2}{Klaster6}
\clearpage
\fg{rys/Klaster/7.png}{Dodanie serwerów dostępu SN1 i SN2}{Klaster7}
\fg{rys/Klaster/8.png}{Ekran Enable Authentication – domyślne ustawienia}{Klaster8}
\clearpage
\fg{rys/Klaster/9.png}{Ekran potwierdzenia ustawień iSCSI}{Klaster9}
\fg{rys/Klaster/10.png}{Ekran z wynikami}{Klaster10}
\clearpage
\fg{rys/Klaster/11.png}{Połączenie z dyskiem na SN1 i SN2}{Klaster11}
\fg{rys/Klaster/12.png}{Automatyczna konfiguracja dysku na SN1 i SN2}{Klaster12}
\clearpage
\fg{rys/Klaster/13.png}{Dodanie serwerów z poziomu SDC06 dla łatwiejszego zarządzania}{Klaster13}
\fg{rys/Klaster/14.png}{Wyszukanie serwerów}{Klaster14}
\clearpage
\fg{rys/Klaster/15.png}{Dodanie SN1 i SN2}{Klaster15}
\fg{rys/Klaster/16.png}{Add roles And Features (SN1--06) z poziomu SDC06}{Klaster16}
\clearpage
\fg{rys/Klaster/17.png}{Instalacja oparta na rolach lub funkcjach}{Klaster17}
\fg{rys/Klaster/18.png}{Wybór serwerów}{Klaster18}
\clearpage
\fg{rys/Klaster/19.png}{Wybranie funcji Failover}{Klaster21}
\fg{rys/Klaster/20.png}{Rozpoczęcie instalacji}{Klaster22}
\clearpage
\fg{rys/Klaster/21.png}{Zakończenie instalacji}{Klaster19}
\fg{rys/Klaster/22.png}{Przestawienie dysku na online w SN1}{Klaster23}
\clearpage
\fg{rys/Klaster/23.png}{inicjalizacja dysku}{Klaster24}
\fg{rys/Klaster/24.png}{Formanie i nadanie etykiety dysku}{Klaster20}
\clearpage
\fg{rys/Klaster/25.png}{Sprawdzenie działania dysku na SN1 i SN2}{Klaster25}
\fg{rys/Klaster/26.png}{Otworzenie kreatora klastra i wybranie SN1 i SN2}{Klaster26}
\clearpage
\fg{rys/Klaster/27.png}{Nazwa klastra i nadanie IP}{Klaster27}
\fg{rys/Klaster/28.png}{Ekran podsumowania}{Klaster28}
\clearpage
\fg{rys/Klaster/29.png}{Ekran przetwarzania klastra}{Klaster29}
\fg{rys/Klaster/30.png}{Ekran podsumowania}{Klaster30}
\clearpage
\fg{rys/Klaster/31.png}{Sprawdzenie działania klastra}{Klaster31}

\subsection{SFS – instalacja i konfiguracja}
Na serwerze \texttt{SDC06} zainstalowano usługę SFS (Scale-Out File Server) w celu udostępnienia zasobów sieciowych dla klastra failover.

\fg{rys/uslugi/SFS/1.png}{Instalacja oparta na rolach lub funkcjach}{SFS1}
\fg{rys/uslugi/SFS/2.png}{Wybór serwerów}{SFS2}
\clearpage
\fg{rys/uslugi/SFS/3.png}{Wybranie funkcji File Server}{SFS3}
\fg{rys/uslugi/SFS/4.png}{Rozpoczęcie instalacji}{SFS4}
\clearpage
\fg{rys/uslugi/SFS/5.png}{Zakończenie instalacji}{SFS5}
\fg{rys/uslugi/SFS/6.png}{Wybór dysku do konfiguracji $\rightarrow$ R: w New iSCSI Virtual Disk}{SFS6}
\clearpage
\fg{rys/uslugi/SFS/7.png}{Wybór nazwy dysku $\rightarrow$ przestrzenFS}{SFS7}
\fg{rys/uslugi/SFS/8.png}{Wybór rozmiaru dysku $\rightarrow$ 10GB}{SFS8}
\clearpage
\fg{rys/uslugi/SFS/9.png}{Nowy cel}{SFS9}
\fg{rys/uslugi/SFS/10.png}{Nazwa celu dyskFS}{SFS10}
\clearpage
\fg{rys/uslugi/SFS/11.png}{Dodanie serwerów dostępu SN1 i SN2}{SFS11}
\fg{rys/uslugi/SFS/12.png}{Ekran potwierdzenia ustawień iSCSI}{SFS12}
\clearpage
\fg{rys/uslugi/SFS/13.png}{Ekran z wynikami}{SFS13}
\fg{rys/uslugi/SFS/14.png}{Pierwszy ekran kreatora roli klastra}{SFS14}
\clearpage
\fg{rys/uslugi/SFS/15.png}{Wybór roli file server}{SFS15}
\fg{rys/uslugi/SFS/16.png}{Wybranie typu FS}{SFS16}
\clearpage
\fg{rys/uslugi/SFS/17.png}{Nazwa serwera i adresu IP}{SFS17}
\fg{rys/uslugi/SFS/18.png}{Przestawienie dysku na online w SN1}{SFS18}
\clearpage
\fg{rys/uslugi/SFS/19.png}{Inicjalizacja dysku}{SFS19}
\fg{rys/uslugi/SFS/20.png}{Formatowanie i nadanie etykiety dysku}{SFS20}
\clearpage
\fg{rys/uslugi/SFS/21.png}{Dodanie dysku do klastra}{SFS21}
\fg{rys/uslugi/SFS/22.png}{Wynik dodania dysku do klastra}{SFS22}
\clearpage
\fg{rys/uslugi/SFS/23.png}{Kontynuacja dodawania roli FS dodanie dysku klastra do roli}{SFS23}
\fg{rys/uslugi/SFS/24.png}{Ekran potwierdzenia}{SFS24}
\clearpage
\fg{rys/uslugi/SFS/25.png}{Podsumowanie dodania roli FS}{SFS25}
\fg{rys/uslugi/SFS/26.png}{Add file share z poziomu Menadżera klastra}{SFS26}
\clearpage
\fg{rys/uslugi/SFS/27.png}{Wybór typu udziału}{SFS27}
\fg{rys/uslugi/SFS/28.png}{Wybór dysku do udziału}{SFS28}
\clearpage
\fg{rys/uslugi/SFS/29.png}{Wybór nazwy udziału – Wspolny}{SFS29}
\fg{rys/uslugi/SFS/30.png}{Wybór ustawień udziału}{SFS30}
\clearpage
\fg{rys/uslugi/SFS/31.png}{Ekran premisji}{SFS31}
\fg{rys/uslugi/SFS/32.png}{Ekran potwierdzenia}{SFS32}
\clearpage
\fg{rys/uslugi/SFS/33.png}{Wynik dodania udziału}{SFS33}
\fg{rys/uslugi/SFS/34.png}{Sprawdzenie działania udziału – cz. 1}{SFS34}
\clearpage
\fg{rys/uslugi/SFS/35.png}{Sprawdzenie działania udziału – cz. 2}{SFS35}
\fg{rys/uslugi/SFS/36.png}{Sprawdzenie działania udziału – cz. 3}{SFS36}
\clearpage
\fg{rys/uslugi/SFS/37.png}{Wybór nazwy udziału – Kadry}{SFS37}
\fg{rys/uslugi/SFS/38.png}{Wybór ustawień udziału}{SFS38}
\clearpage
\fg{rys/uslugi/SFS/39.png}{Usunięcie premisji}{SFS39}
\fg{rys/uslugi/SFS/40.png}{Nowe premisje}{SFS40}
\clearpage
\fg{rys/uslugi/SFS/41.png}{Wynik dodania udziału}{SFS41}
\fg{rys/uslugi/SFS/42.png}{Sprawdzenie działania udziału – Kadry}{SFS42}
\clearpage
\fg{rys/uslugi/SFS/43.png}{Wybór nazwy udziału – Ksiegowosc}{SFS43}
\fg{rys/uslugi/SFS/44.png}{Konwersja dziedziczonych premisji na bezpośrednie}{SFS44}
\clearpage
\fg{rys/uslugi/SFS/45.png}{Usunięcie dziedziczonych premisji}{SFS45}
\fg{rys/uslugi/SFS/46.png}{Nowe premisje}{SFS46}
\clearpage
\fg{rys/uslugi/SFS/47.png}{Sprawdzenie działania udziału – Księgowość}{SFS47}
\fg{rys/uslugi/SFS/48.png}{Wybór nazwy udziału – Marketing}{SFS48}
\clearpage
\fg{rys/uslugi/SFS/49.png}{Nowe premisje}{SFS49}
\fg{rys/uslugi/SFS/50.png}{Wynik dodania udziału}{SFS50}
\clearpage
\fg{rys/uslugi/SFS/51.png}{Wybór nazwy udziału – Programista}{SFS51}
\fg{rys/uslugi/SFS/52.png}{Nowe premisje}{SFS52}
\clearpage
\fg{rys/uslugi/SFS/53.png}{Wynik dodania udziału}{SFS53}
\fg{rys/uslugi/SFS/54.png}{Wybór nazwy udziału – IT}{SFS54}
\clearpage
\fg{rys/uslugi/SFS/55.png}{Nowe premisje}{SFS55}
\fg{rys/uslugi/SFS/56.png}{Wynik dodania udziału}{SFS56}

\subsection{Group Policy Objects (GPO)}
Na serwerze \texttt{SDC06} skonfigurowano polityki Group Policy Objects (GPO) w celu centralnego zarządzania ustawieniami komputerów oraz użytkowników w organizacji. Dzięki temu administratorzy mogą automatycznie wdrażać zasady bezpieczeństwa, konfigurację systemów operacyjnych oraz ustawienia aplikacji w sieci, eliminując potrzebę ręcznego konfigurowania każdego urządzenia. GPO zapewnia jednolitość w zarządzaniu urządzeniami,
\subsubsection{Polisy dotyczące dysków}
\fg{rys/uslugi/GPO/1.png}{Otwarcie Group Policy Management}{GPO1}
\fg{rys/uslugi/GPO/2.png}{Utworzenie nowej polityki}{GPO2}
\clearpage
\fg{rys/uslugi/GPO/3.png}{Wybranie Drive Maps $\rightarrow$ New $\rightarrow$ Mapped Drive}{GPO3}
\fg{rys/uslugi/GPO/4.png}{Konfiguracja dysku – Wspolny}{GPO4}
\clearpage
\fg{rys/uslugi/GPO/5.png}{Test dysku – Wspolny}{GPO5}
\fg{rys/uslugi/GPO/6.png}{Edycja dysku dla programisty}{GPO6}
\clearpage
\fg{rys/uslugi/GPO/7.png}{Wybranie Drive Maps $\rightarrow$ New $\rightarrow$ Mapped Drive}{GPO7}
\fg{rys/uslugi/GPO/8.png}{Konfiguracja dysku – cz. 1}{GPO8}
\clearpage
\fg{rys/uslugi/GPO/9.png}{Konfiguracja dysku – cz. 2}{GPO9}
\fg{rys/uslugi/GPO/10.png}{Konfiguracja dysku – cz. 3}{GPO10}
\clearpage
\fg{rys/uslugi/GPO/11.png}{Konfiguracja dysku – cz. 4}{GPO11}
\fg{rys/uslugi/GPO/12.png}{Konfiguracja dysku – cz. 5}{GPO12}
\clearpage
\fg{rys/uslugi/GPO/13.png}{Test dysku}{GPO13}
\fg{rys/uslugi/GPO/14.png}{Konfiguracja dysku dla kadr – cz. 1}{GPO14}
\clearpage
\fg{rys/uslugi/GPO/15.png}{Konfiguracja dysku dla kadr – cz. 2}{GPO15}
\fg{rys/uslugi/GPO/16.png}{Konfiguracja dysku dla kadr – cz. 3}{GPO16}
\clearpage
\fg{rys/uslugi/GPO/17.png}{Konfiguracja dysku dla ksiegowosc – cz. 1}{GPO17}
\fg{rys/uslugi/GPO/18.png}{Konfiguracja dysku dla ksiegowosc – cz. 2}{GPO18}
\clearpage
\fg{rys/uslugi/GPO/19.png}{Konfiguracja dysku dla ksiegowosc – cz. 3}{GPO19}
\fg{rys/uslugi/GPO/20.png}{Konfiguracja dysku dla Marketing – cz. 1}{GPO20}
\clearpage
\fg{rys/uslugi/GPO/21.png}{Konfiguracja dysku dla Marketing – cz. 2}{GPO21}
\fg{rys/uslugi/GPO/22.png}{Konfiguracja dysku dla Marketing – cz. 3}{GPO22}
\clearpage
\fg{rys/uslugi/GPO/23.png}{Konfiguracja dysku dla IT – cz. 1}{GPO23}
\fg{rys/uslugi/GPO/24.png}{Konfiguracja dysku dla IT – cz. 2}{GPO24}
\clearpage
\fg{rys/uslugi/GPO/25.png}{Konfiguracja dysku dla IT – cz. 3}{GPO25}

\subsubsection{Zmienna środowiskowa i tworzenie folderu}
\fg{rys/uslugi/GPO/26.png}{Utworzenie nowej polityki}{GPO26}
\clearpage
\fg{rys/uslugi/GPO/27.png}{Nazwa polityki}{GPO27}
\fg{rys/uslugi/GPO/28.png}{Edycja polityki}{GPO28}
\clearpage

\fg{rys/uslugi/GPO/29.png}{Wybranie User Configuration $\rightarrow$ Preferences $\rightarrow$ Windows Settings $\rightarrow$ Environment $\rightarrow$ New $\rightarrow$ Environment Variable}{GPO29}
\fg{rys/uslugi/GPO/30.png}{Konfiguracja zmiennej środowiskowej}{GPO30}
\clearpage
\fg{rys/uslugi/GPO/31.png}{Test zmiennej środowiskowej}{GPO31}
\fg{rys/uslugi/GPO/32.png}{Utworzenie nowej polityki}{GPO26}
\clearpage
\fg{rys/uslugi/GPO/33.png}{Nazwa polityki}{GPO27}
\fg{rys/uslugi/GPO/34.png}{Edycja polityki}{GPO28}
\clearpage
\fg{rys/uslugi/GPO/35.png}{Wybranie User Configuration $\rightarrow$ Preferences $\rightarrow$ Windows Settings $\rightarrow$ Folders $\rightarrow$ New $\rightarrow$ Folder}{GPO29}
\fg{rys/uslugi/GPO/36.png}{Konfiguracja folderu}{GPO30}
\clearpage
\fg{rys/uslugi/GPO/37.png}{Stan przed zastosowaniem polityki}{GPO31}
\fg{rys/uslugi/GPO/38.png}{Zastosowanie polityki}{GPO31}
\clearpage
\fg{rys/uslugi/GPO/39.png}{Stan po zastosowaniu polityki}{GPO31}


\subsection{Instalacja usługi Dynamic Host Configuration Protocol (DHCP)}
Na klastrze zainstalowano usługę Dynamic Host Configuration Protocol (DHCP) w celu automatycznego przydzielania adresów IP oraz konfiguracji sieci dla urządzeń w sieci lokalnej. 
\fg{rys/uslugi/DHCP/1.png}{Otwarcie kreatora instalacji roli na klastrze}{DHCP1}
\fg{rys/uslugi/DHCP/2.png}{Pierwsza strona kreatora}{DHCP2}
\clearpage
\fg{rys/uslugi/DHCP/3.png}{Instalacja oparta na rolach lub funkcjach}{DHCP3}
\fg{rys/uslugi/DHCP/4.png}{Wybór serwerów}{DHCP4}
\clearpage
\fg{rys/uslugi/DHCP/5.png}{Wybranie funkcji DHCP}{DHCP5}
\fg{rys/uslugi/DHCP/6.png}{Ekran dotyczący DHCP Server}{DHCP6}
\clearpage
\fg{rys/uslugi/DHCP/7.png}{Rozpoczęcie instalacji}{DHCP7}
\fg{rys/uslugi/DHCP/8.png}{Powrót do Failover Cluster Manager i kreatora funkcji DHCP – nadanie nazwy serwera i ustalenie adresu IP}{DHCP8}
\clearpage
\fg{rys/uslugi/DHCP/9.png}{Dodanie dysku 5GB do maszyny wirtualnej}{DHCP9}
\fg{rys/uslugi/DHCP/10.png}{Widok dysku w ustawieniach maszyny wirtualnej}{DHCP10}
\clearpage
\fg{rys/uslugi/DHCP/11.png}{Initializacja dysku}{DHCP11}
\fg{rys/uslugi/DHCP/12.png}{Formatowanie dysku i nadanie etykiety DHCP}{DHCP12}
\clearpage
\fg{rys/uslugi/DHCP/13.png}{Przejście do New iSCSI Virtual Disk}{DHCP13}
\fg{rys/uslugi/DHCP/14.png}{Dokończenie konfiguracji roli DHCP na SN1 i SN2}{DHCP14}
\clearpage
\fg{rys/uslugi/DHCP/15.png}{Wybór dysku do konfiguracji $\rightarrow$ E: w New iSCSI Virtual Disk}{DHCP15}
\fg{rys/uslugi/DHCP/16.png}{Wybór nazwy dysku $\rightarrow$ DHCP}{DHCP16}
\clearpage
\fg{rys/uslugi/DHCP/17.png}{Wybór rozmiaru dysku $\rightarrow$ 5GB}{DHCP17}
\fg{rys/uslugi/DHCP/18.png}{Nowy cel}{DHCP18}
\clearpage
\fg{rys/uslugi/DHCP/19.png}{Nazwa celu DHCP-SN1-SN2-KLASTER}{DHCP19}
\fg{rys/uslugi/DHCP/20.png}{Dodanie serwerów dostępu SN1 i SN2}{DHCP20}
\clearpage
\fg{rys/uslugi/DHCP/21.png}{Ekran potwierdzenia ustawień iSCSI}{DHCP21}
\fg{rys/uslugi/DHCP/22.png}{Połaćzenie z dyskiem na SN1 i SN2}{DHCP22}
\clearpage
\fg{rys/uslugi/DHCP/23.png}{Automatyczna konfiguracja dysku na SN1 i SN2}{DHCP23}
\fg{rys/uslugi/DHCP/24.png}{Przestawienie dysków na online na SN1}{DHCP24}
\clearpage
\fg{rys/uslugi/DHCP/25.png}{Inicjalizacja dysku na SN1}{DHCP25}
\fg{rys/uslugi/DHCP/26.png}{Formatowanie i nadanie etykiety dysku na SN1}{DHCP26}
\clearpage
\fg{rys/uslugi/DHCP/27.png}{Dodanie dysku do klastra}{DHCP27}
\fg{rys/uslugi/DHCP/28.png}{Wynik dodania dysku do klastra}{DHCP28}
\clearpage
\fg{rys/uslugi/DHCP/29.png}{Kontynuacja dodawania roli DHCP dodanie dysku klastra do roli}{DHCP29}
\fg{rys/uslugi/DHCP/30.png}{Ekran potwierdzenia}{DHCP30}
\clearpage
\fg{rys/uslugi/DHCP/31.png}{Podsumowanie dodania roli DHCP}{DHCP31}
\fg{rys/uslugi/DHCP/32.png}{Otworzenie zarządzania DHCP}{DHCP32}
\clearpage
\fg{rys/uslugi/DHCP/33.png}{Dodanie nowego zakresu}{DHCP33}
\fg{rys/uslugi/DHCP/34.png}{Konfiguracja zakresu – nazwa}{DHCP34}
\clearpage
\fg{rys/uslugi/DHCP/35.png}{Konfiguracja zakresu – zakres adresów}{DHCP35}
\fg{rys/uslugi/DHCP/36.png}{Konfiguracja wykluczeń}{DHCP36}
\clearpage
\fg{rys/uslugi/DHCP/37.png}{Konfiguracja zakresu – długość dzierżawy}{DHCP37}
\fg{rys/uslugi/DHCP/38.png}{Potwierdzenie konfiguracji opcji DHCP}{DHCP38}
\clearpage
\fg{rys/uslugi/DHCP/39.png}{Konfiguracja zakresu – brama domyślna}{DHCP39}
\fg{rys/uslugi/DHCP/40.png}{Konfiguracja zakresu – serwer DNS}{DHCP40}
\clearpage
\fg{rys/uslugi/DHCP/41.png}{Konfiguracja zakresu – serwer WINS}{DHCP41}
\fg{rys/uslugi/DHCP/42.png}{Aktywacja zakresu}{DHCP42}
\clearpage
\fg{rys/uslugi/DHCP/43.png}{Sprawdzenie działania DHCP – cz. 1}{DHCP43}
\fg{rys/uslugi/DHCP/44.png}{Sprawdzenie działania DHCP – cz. 2}{DHCP44}


\subsection{Instalacja usługi Windows Deployment Services (WDS)}
\fg{rys/uslugi/WDS/1.png}{Instalacja oparta na rolach lub funkcjach}{WDS1}
\fg{rys/uslugi/WDS/2.png}{Wybór serwera SDC06}{WDS2}
\clearpage
\fg{rys/uslugi/WDS/3.png}{Wybranie funkcji Windows Deployment Services}{WDS3}
\fg{rys/uslugi/WDS/4.png}{Role Services – wybór obu}{WDS4}
\clearpage
\fg{rys/uslugi/WDS/5.png}{Rozpoczęcie instalacji}{WDS5}
\fg{rys/uslugi/WDS/6.png}{Instalacja zakończona sukcesem}{WDS6}
\clearpage
\fg{rys/uslugi/WDS/7.png}{Otwarcie Windows Deployment Services Console}{WDS7}
\fg{rys/uslugi/WDS/8.png}{Konfiguracja serwera}{WDS8}
\clearpage
\fg{rys/uslugi/WDS/9.png}{Zintegrowanie WDS z Active Directory}{WDS9}
\fg{rys/uslugi/WDS/10.png}{Wybór ścieżki do magazynu obrazów}{WDS10}
\clearpage
\fg{rys/uslugi/WDS/11.png}{Odpowaadanie wszystkim hostom}{WDS11}
\fg{rys/uslugi/WDS/12.png}{Dodanie obrazu instalacyjnego}{WDS12}
\clearpage
\fg{rys/uslugi/WDS/13.png}{Utworzenie katalogu ISO na dysku C\: i przerzucenie pliku ISO do maszyny wirtualnej}{WDS13}
\fg{rys/uslugi/WDS/14.png}{Dodanie obrazu instalacyjnego}{WDS14}
\clearpage
\fg{rys/uslugi/WDS/15.png}{Stworzenie nowej grupy}{WDS15}
\fg{rys/uslugi/WDS/16.png}{Zamontoawnie obrazu}{WDS16}
\clearpage
\fg{rys/uslugi/WDS/21.png}{Skopiowanie plików z obrazu do katalogu C:\textbackslash sources}{WDS17}
\fg{rys/uslugi/WDS/18.png}{Wybór obrazu instalacyjnego}{WDS18}
\clearpage
\fg{rys/uslugi/WDS/19.png}{Przeklikanie kreatora do końca}{WDS19}
\fg{rys/uslugi/WDS/20.png}{Dodanie obrazu bootowalnego}{WDS20}
\clearpage
\fg{rys/uslugi/WDS/21.png}{Wybór obrazu boot.wim}{WDS21}
\fg{rys/uslugi/WDS/22.png}{Przeklikanie kreatora do końca}{WDS22}
\clearpage
\fg{rys/uslugi/WDS/23.png}{Stworzenie nowej maszyny wirtualnej}{WDS23}
\fg{rys/uslugi/WDS/24.png}{Stworzenie nowej maszyny wirtualnej – konfiguracja}{WDS24}
\clearpage
\fg{rys/uslugi/WDS/25.png}{Stworzenie nowej maszyny wirtualnej – wybór wewnętrznej sieci}{WDS25}
\fg{rys/uslugi/WDS/26.png}{Uruchomienie maszyny wirtualnej}{WDS26}
\clearpage
\fg{rys/uslugi/WDS/27.png}{Uruchomienie maszyny wirtualnej – załadowanie obrazu przez sieć}{WDS27}



\subsection{Instalacja serwera wydruku}
\fg{rys/uslugi/Print/1.png}{Instalacja oparta na rolach lub funkcjach}{Print1}
\fg{rys/uslugi/Print/2.png}{Wybór serwera SDC06}{Print2}
\clearpage
\fg{rys/uslugi/Print/3.png}{Wybranie funkcji Print and Document Services}{Print3}
\fg{rys/uslugi/Print/4.png}{Role Services – wybór Print Server}{Print4}
\clearpage
\fg{rys/uslugi/Print/5.png}{Rozpoczęcie instalacji}{Print5}
\fg{rys/uslugi/Print/6.png}{Potwierdzenie instalacji}{Print6}
\clearpage
\fg{rys/uslugi/Print/7.png}{Instalacja sterowników drukarki}{Print7}
\fg{rys/uslugi/Print/8.png}{Otwarcie Print Management}{Print8}
\clearpage
\fg{rys/uslugi/Print/9.png}{Dodanie drukarki}{Print9}
\fg{rys/uslugi/Print/10.png}{Wybór drukarki sieciowej}{Print10}
\clearpage
\fg{rys/uslugi/Print/11.png}{Wybór drukarki sieciowej – wpisanie adresu IP}{Print11}
\fg{rys/uslugi/Print/12.png}{Udostępnienie drukarki sieciowej przy jej dodawaniu}{Print12}
\clearpage
\fg{rys/uslugi/Print/13.png}{Ekran podsumowania}{Print13}
\fg{rys/uslugi/Print/14.png}{Zakończenie dodawania drukarki i wydruk strony testowej}{Print14}
\clearpage
\fg{rys/uslugi/Print/15.png}{Wejście w właściwości drukarki – zakładka sharing}{Print15}
\fg{rys/uslugi/Print/16.png}{Dodanie drukarki na stacji klienckiej}{Print16}
\clearpage
\fg{rys/uslugi/Print/17.png}{Pomyślne dodanie drukarki}{Print17}

\subsection{Instalacja serwera WWW}
\fg{rys/uslugi/IIS/1.png}{Instalacja oparta na rolach lub funkcjach}{WWW1}
\fg{rys/uslugi/IIS/2.png}{Wybór serwera SDC06}{WWW2}
\clearpage
\fg{rys/uslugi/IIS/3.png}{Wybranie funkcji Web Server (IIS)}{WWW3}
\fg{rys/uslugi/IIS/4.png}{Role Services – domyślne ustawienia}{WWW4}
\clearpage
\fg{rys/uslugi/IIS/5.png}{Ekran potwierdzenia}{WWW5}
\fg{rys/uslugi/IIS/6.png}{Instalacja zakończona sukcesem}{WWW6}
\clearpage
\fg{rys/uslugi/IIS/7.png}{Stworzenie katalogu C:\textbackslash WWW}{WWW7}
\fg{rys/uslugi/IIS/8.png}{Dodanie strony }{WWW8}
\clearpage
\fg{rys/uslugi/IIS/9.png}{Wybór strony – nazwa, ścieżka, host}{WWW9}
\fg{rys/uslugi/IIS/10.png}{Utworzenie pliku index.html}{WWW11}
\clearpage
\fg{rys/uslugi/IIS/11.png}{Weryfikacja działania strony}{WWW12}
\fg{rys/uslugi/IIS/12.png}{Usunięcie domyślnej strony}{WWW13}
\clearpage
\fg{rys/uslugi/IIS/13.png}{Sprawdzenie działania strony pod adresem binarybuilders.ad}{WWW14}
\fg{rys/uslugi/IIS/14.png}{Dodanie bidingu dla strony binarybuilders.ad}{WWW15}
\clearpage
\fg{rys/uslugi/IIS/15.png}{Sprawdzenie działania strony pod adresem binarybuilders.ad}{WWW14}




\subsection{Instalacja WordPress}

