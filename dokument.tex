%╔════════════════════════════╗
%║	  Szablon dostosował	  ║
%║	mgr inż. Dawid Kotlarski  ║
%║		  06.10.2024		  ║
%╚════════════════════════════╝
\documentclass[12pt,twoside,a4paper,openany]{article}

    % ------------------------------------------------------------------------
% PAKIETY
% ------------------------------------------------------------------------

%różne pakiety matematyczne, warto przejrzeć dokumentację, muszą być powyżej ustawień językowych.
\usepackage{mathrsfs}   %Różne symbole matematyczne opisane w katalogu ~\doc\latex\comprehensive. Zamienia \mathcal{L} ze zwykłego L na L-transformatę.
\usepackage{eucal}      %Różne symbole matematyczne.
\usepackage{amssymb}    %Różne symbole matematyczne.
\usepackage{amsmath}    %Dodatkowe funkcje matematyczne, np. polecenie \dfac{}{} skladajace ulamek w trybie wystawionym (porównaj $\dfrac{1}{2}$, a $\frac{1}{2}$).

%język polski i klawiatura
\usepackage[polish]{babel}
\usepackage{csquotes}
%\usepackage{qtimes} % czcionka Times new Roman
\usepackage{polski}

\usepackage{ifluatex}

\ifluatex
  %czcionka
  \usepackage{fontspec}
  \setmainfont{Calibri}

  %obsługa pdf'a
  \usepackage[luatex,usenames,dvipsnames]{color}      %Obsługa kolorów. Opcje usenames i dvipsnames wprowadzają dodatkowe nazwy kolorow.
  \usepackage[luatex,pagebackref=false,draft=false,pdfpagelabels=false,colorlinks=true,urlcolor=cyan,linkcolor=blue,filecolor=magenta,citecolor=green,pdfstartview=FitH,pdfstartpage=1,pdfpagemode=UseOutlines,bookmarks=true,bookmarksopen=true,bookmarksopenlevel=2,bookmarksnumbered=true,pdfauthor={Dawid Kotlarski},pdftitle={Dokumentacja Projektowa},pdfsubject={},pdfkeywords={transient recovery voltage trv},unicode=true]{hyperref}   %Opcja pagebackref=true dotyczy bibliografii: pokazuje w spisie literatury numery stron, na których odwołano się do danej pozycji.
\else
  \usepackage[pdftex,usenames,dvipsnames]{color}      %Obsługa kolorów. Opcje usenames i dvipsnames wprowadzają dodatkowe nazwy kolorow.
\usepackage[pdftex,pagebackref=false,draft=false,pdfpagelabels=false,colorlinks=true,urlcolor=blue,linkcolor=black,citecolor=green,pdfstartview=FitH,pdfstartpage=1,pdfpagemode=UseOutlines,bookmarks=true,bookmarksopen=true,bookmarksopenlevel=2,bookmarksnumbered=true,pdfauthor={Dawid Kotlarski},pdftitle={Dokumentacja Projektowa},pdfsubject={},pdfkeywords={transient recovery voltage trv},unicode=true]{hyperref}  %Opcja pagebackref=true dotyczy bibliografii: pokazuje w spisie literatury numery stron, na których odwołano się do danej pozycji.
\fi

%bibliografia
%\usepackage[numbers,sort&compress]{natbib}  %Porządkuje zawartość odnośników do literatury, np. [2-4,6]. Musi być pod pdf'em, a styl bibliogfafii musi mieć nazwę z dodatkiem 'nat', np. \bibliographystyle{unsrtnat} (w kolejności cytowania).
\usepackage[
  backend=biber,
  style=numeric,
  sorting=none
]{biblatex}
\addbibresource{bibliografia.bib}
\usepackage{hypernat}                       %Potrzebna pakietowi natbib do wspolpracy z pakietem hyperref (wazna kolejnosc: 1. hyperref, 2. natbib, 3. hypernat).

%grafika i geometria strony
\usepackage{extsizes}           %Dostepne inne rozmiary czcionek, np. 14 w poleceniu: \documentclass[14pt]{article}.
\usepackage[final]{graphicx}
\usepackage[a4paper,left=3.5cm,right=2.5cm,top=2.5cm,bottom=2.5cm]{geometry}

%strona tytułowa
\usepackage{strona_tytulowa}

%inne
% \usepackage{lastpage} %! do numerowania stron w formacie (x z y)
\usepackage[hide]{todo}                     %Wprowadza polecenie \todo{treść}. Opcje pakietu: hide/show. Polecenie \todos ma byc na koncu dokumentu, wszystkie \todo{} po \todos sa ignorowane.
\usepackage[basic,physics]{circ}            %Wprowadza środowisko circuit do rysowania obwodów elektrycznych. Musi byc poniżej pakietow językowych.
\usepackage[sf,bf,outermarks]{titlesec}     %Troszczy się o wygląd tytułów rozdziałów (section, subsection, ...). sf oznacza czcionkę sans serif (typu arial), bf -- bold. U mnie: oddzielna linia dla naglowku paragraph. Patrz tez: tocloft -- lepiej robi format spisu tresci.
\usepackage{tocloft}                        %Troszczy się o format spisu trsci.
\usepackage{expdlist}    %Zmienia definicję środowiska description, daje większe możliwości wpływu na wygląd listy.
\usepackage{flafter}     %Wprowadza parametr [tb] do polecenia \suppressfloats[t] (polecenie to powoduje nie umieszczanie rysunkow, tabel itp. na stronach, na ktorych jest to polecenie (np. moze byc to stroma z tytulem rozdzialu, ktory chcemy zeby byl u samej gory, a nie np. pod rysunkiem)).
\usepackage{array}       %Ładniej drukuje tabelki (np. daje wiecej miejsca w komorkach -- nie są tak ścieśnione, jak bez tego pakietu).
\usepackage{listings}    %Listingi programow.
\usepackage[format=hang,labelsep=period,labelfont={bf,small},textfont=small]{caption}   %Formatuje podpisy pod rysunkami i tabelami. Parametr 'hang' powoduje wcięcie kolejnych linii podpisu na szerokosc nazwy podpisu, np. 'Rysunek 1.'.
\usepackage{appendix}    %Troszczy się o załączniki.
\usepackage{floatflt}    %Troszczy się o oblewanie rysunkow tekstem.
\usepackage{here}        %Wprowadza dodtkowy parametr umiejscowienia rysunków, tabel, itp.: H (duże). Umiejscawia obiekty ruchome dokladnie tam gdzie są w kodzie źródłowym dokumentu.
\usepackage{makeidx}     %Troszczy się o indeks (skorowidz).

%nieużywane, ale potencjalnie przydatne
\usepackage{sectsty}           %Formatuje nagłówki, np. żeby były kolorowe -- polecenie: \allsectionsfont{\color{Blue}}.
%\usepackage{version}           %Wersje dokumentu.

%============
\usepackage{longtable}			%tabelka
\usepackage{tabularx}
%============

%============
% Ustawienia listingów do kodu
%============

\usepackage{listings}
\usepackage{xcolor}

\definecolor{codegreen}{rgb}{0,0.6,0}
\definecolor{codegray}{rgb}{0.5,0.5,0.5}
\definecolor{codepurple}{rgb}{0.58,0,0.82}
\definecolor{backcolour}{rgb}{0.95,0.95,0.92}

% Definicja stylu "mystyle"
\lstdefinestyle{mystyle}{
backgroundcolor=\color{backcolour},
commentstyle=\color{codegreen},
keywordstyle=\color{blue},	%magenta
numberstyle=\tiny\color{codegray},
stringstyle=\color{codepurple},
basicstyle=\ttfamily\footnotesize,
breakatwhitespace=false,
breaklines=true,
captionpos=b,
keepspaces=true,
numbers=left,
numbersep=5pt,
showspaces=false,
showstringspaces=false,
showtabs=false,
tabsize=2,
literate=
  {á}{{\'a}}1 {é}{{\'e}}1 {í}{{\'i}}1 {ó}{{\'o}}1 {ú}{{\'u}}1
{Á}{{\'A}}1 {É}{{\'E}}1 {Í}{{\'I}}1 {Ó}{{\'O}}1 {Ú}{{\'U}}1
{à}{{\`a}}1 {è}{{\`e}}1 {ì}{{\`i}}1 {ò}{{\`o}}1 {ù}{{\`u}}1
{À}{{\`A}}1 {È}{{\`E}}1 {Ì}{{\`I}}1 {Ò}{{\`O}}1 {Ù}{{\`U}}1
{ä}{{\"a}}1 {ë}{{\"e}}1 {ï}{{\"i}}1 {ö}{{\"o}}1 {ü}{{\"u}}1
{Ä}{{\"A}}1 {Ë}{{\"E}}1 {Ï}{{\"I}}1 {Ö}{{\"O}}1 {Ü}{{\"U}}1
{â}{{\^a}}1 {ê}{{\^e}}1 {î}{{\^i}}1 {ô}{{\^o}}1 {û}{{\^u}}1
{Â}{{\^A}}1 {Ê}{{\^E}}1 {Î}{{\^I}}1 {Ô}{{\^O}}1 {Û}{{\^U}}1
{ã}{{\~a}}1 {ẽ}{{\~e}}1 {ĩ}{{\~i}}1 {õ}{{\~o}}1 {ũ}{{\~u}}1
{Ã}{{\~A}}1 {Ẽ}{{\~E}}1 {Ĩ}{{\~I}}1 {Õ}{{\~O}}1 {Ũ}{{\~U}}1
{œ}{{\oe}}1 {Œ}{{\OE}}1 {æ}{{\ae}}1 {Æ}{{\AE}}1 {ß}{{\ss}}1
{ű}{{\H{u}}}1 {Ű}{{\H{U}}}1 {ő}{{\H{o}}}1 {Ő}{{\H{O}}}1
{ç}{{\c c}}1 {Ç}{{\c C}}1 {ø}{{\o}}1 {Ø}{{\O}}1 {å}{{\r a}}1 {Å}{{\r A}}1
{€}{{\euro}}1 {£}{{\pounds}}1 {«}{{\guillemotleft}}1
{»}{{\guillemotright}}1 {ñ}{{\~n}}1 {Ñ}{{\~N}}1 {¿}{{?`}}1 {¡}{{!`}}1
{ą}{{\k{a}}}1 {ć}{{\'{c}}}1 {ę}{{\k{e}}}1 {ł}{{\l}}1 {ń}{{\'n}}1
{ó}{{\'o}}1 {ś}{{\'s}}1 {ź}{{\'z}}1 {ż}{{\.{z}}}1
{Ą}{{\k{A}}}1 {Ć}{{\'{C}}}1 {Ę}{{\k{E}}}1 {Ł}{{\L}}1 {Ń}{{\'N}}1
{Ó}{{\'O}}1 {Ś}{{\'S}}1 {Ź}{{\'Z}}1 {Ż}{{\.{Z}}}1
}

\lstset{style=mystyle} % Deklaracja aktywnego stylu
%===========

%PAGINA GÓRNA I DOLNA
\usepackage{fancyhdr}          %Dodaje naglowki jakie się chce.
\pagestyle{fancy}
\fancyhf{}
% numery stron na środku dolnej stopki
\fancyfoot[C]{\footnotesize DOKUMENTACJA PROJEKTOWA – SYSTEMY OPERACYJNE  \\
  \normalsize\sffamily  \thepage\ z~\pageref{LastPage}}

%\fancyhead[L]{\small\sffamily \nouppercase{\leftmark}}
\fancyhead[C]{\footnotesize \textit{AKADEMIA NAUK STOSOWANYCH W NOWYM SĄCZU}\\}

\renewcommand{\headrulewidth}{0.4pt}
\renewcommand{\footrulewidth}{0.4pt}

    % ------------------------------------------------------------------------
% USTAWIENIA
% ------------------------------------------------------------------------

% ------------------------------------------------------------------------
%   Kropki po numerach sekcji, podsekcji, itd.
%   Np. 1.2. Tytuł podrozdziału
% ------------------------------------------------------------------------
\makeatletter
    \def\numberline#1{\hb@xt@\@tempdima{#1.\hfil}}                      %kropki w spisie treści
    \renewcommand*\@seccntformat[1]{\csname the#1\endcsname.\enspace}   %kropki w treści dokumentu
\makeatother

% ------------------------------------------------------------------------
%   Numeracja równań, rysunków i tabel
%   Np.: (1.2), gdzie:
%   1 - numer sekcji, 2 - numer równania, rysunku, tabeli
%   Uwaga ogólna: o otoczeniu figure ma być najpierw \caption{}, potem \label{}, inaczej odnośnik nie działa!
% ------------------------------------------------------------------------
\makeatletter
    \@addtoreset{equation}{section} %resetuje licznik po rozpoczęciu nowej sekcji
    \renewcommand{\theequation}{{\thesection}.\@arabic\c@equation} %dodaje kropki

    \@addtoreset{figure}{section}
    \renewcommand{\thefigure}{{\thesection}.\@arabic\c@figure}

    \@addtoreset{table}{section}
    \renewcommand{\thetable}{{\thesection}.\@arabic\c@table}
\makeatother

% ------------------------------------------------------------------------
% Tablica
% ------------------------------------------------------------------------
\newenvironment{tabela}[3]
{
    \begin{table}[!htb]
    \centering
    \caption[#1]{#2}
    \vskip 9pt
    #3
}{
    \end{table}
}

% ------------------------------------------------------------------------
% Dostosowanie wyglądu pozycji listy \todos, np. zamiast 'p.' jest 'str.'
% ------------------------------------------------------------------------
\renewcommand{\todoitem}[2]{%
    \item \label{todo:\thetodo}%
    \ifx#1\todomark%
        \else\textbf{#1 }%
    \fi%
    (str.~\pageref{todopage:\thetodo})\ #2}
\renewcommand{\todoname}{Do zrobienia...}
\renewcommand{\todomark}{~uzupełnić}

% ------------------------------------------------------------------------
% Definicje
% ------------------------------------------------------------------------
\def\nonumsection#1{%
    \section*{#1}%
    \addcontentsline{toc}{section}{#1}%
    }
\def\nonumsubsection#1{%
    \subsection*{#1}%
    \addcontentsline{toc}{subsection}{#1}%
    }
\reversemarginpar %umieszcza notki po lewej stronie, czyli tam gdzie jest więcej miejsca
\def\notka#1{%
    \marginpar{\footnotesize{#1}}%
    }
\def\mathcal#1{%
    \mathscr{#1}%
    }
\newcommand{\atp}{ATP/EMTP} % Inaczej: \def\atp{ATP/EMTP}

% ------------------------------------------------------------------------
% Inne
% ------------------------------------------------------------------------
\frenchspacing                      
\hyphenation{ATP/-EMTP}             %dzielenie wyrazu w danym miejscu
\setlength{\parskip}{3pt}           %odstęp pomiędzy akapitami
\linespread{1.3}                    %odstęp pomiędzy liniami (interlinia)
\setcounter{tocdepth}{4}            %uwzględnianie w spisie treści czterech poziomów sekcji
\setcounter{secnumdepth}{4}         %numerowanie do czwartego poziomu sekcji 
\titleformat{\paragraph}[hang]      %wygląd nagłówków
{\normalfont\sffamily\bfseries}{\theparagraph}{1em}{}

%komenda do łatwiejszego wstawiania zdjęć
\newcommand*{\fg}[4][\textwidth]{
    \begin{figure}[!htb]
        \begin{center}
            \includegraphics[width=#1]{#2}
            \caption{#3}
            \label{rys:#4}
        \end{center}
    \end{figure}
}

\newcommand*{\Oznacz}[2]{
\ref{#1:#2} (s. \pageref{#1:#2})
}

\newcommand*{\OznaczZdjecie}[2][Rysunek]{
#1 \Oznacz{rys}{#2}
}
    
\newcommand*{\OznaczKod}[1]{
\Oznacz{lst}{#1}
}

\newcommand*{\ListingFile}[2]{
    \lstinputlisting[caption=#1, label={lst:#2}, language=C++]{kod/#2.txt}
}


    %polecenia zdefiniowane w pakiecie strona_tytulowa.sty
    \title{Projektowanie i wdrażanie systemu informatycznego dla przedsiębiorstwa zgodnie z określonymi założeniami}		%...Wpisać nazwę projektu...
    \author{Imie Nazwisko}
    \authorI{}
    \authorII{}		%jeśli są dwie osoby w projekcie to zostawiamy:    \authorII{}
		
	\uczelnia{AKADEMIA NAUK STOSOWANYCH \\W NOWYM SĄCZU}
    \instytut{Wydział Nauk Inżynieryjnych}
    \kierunek{Katedra Informatyki}
    \praca{DOKUMENTACJA PROJEKTOWA}
    \przedmiot{SYSTEMY OPERACYJNE}
    \prowadzacy{mgr inż. Jan Kozieński}
    \rok{2025}

%definicja składni mikrotik
\usepackage{fancyvrb}
\DefineVerbatimEnvironment{MT}{Verbatim}%
{commandchars=\+\[\],fontsize=\small,formatcom=\color{red},frame=lines,baselinestretch=1,} 
\let\mt\verb
%zakonczenie definicji składni mikrotik

\usepackage{fancyhdr}    %biblioteka do nagłówka i stopki
\begin{document}

\renewcommand{\figurename}{Rys.}    %musi byc pod \begin{document}, bo w~tym miejscu pakiet 'babel' narzuca swoje ustawienia
\renewcommand{\tablename}{Tab.}     %j.w.
\thispagestyle{empty}               %na tej stronie: brak numeru
\stronatytulowa                     %strona tytułowa tworzona przez pakiet strona_tytulowa.tex

\newpage

\begin{table}[]
  \centering
  \resizebox{\columnwidth}{!}{%
    \begin{tabular}{|p{0.5\textwidth}|p{0.5\textwidth}|}
      \hline
      \multicolumn{2}{|c|}{\textbf{\begin{tabular}[c]{@{}c@{}}\large{ANS Nowy Sącz}\\ \large{Systemy operacyjne - projekt}\\ \\ Studia stacjonarne\\ Semestr zimowy 2024 / 2025\end{tabular}}}                                                 \\ \hline
      \multicolumn{2}{|p{\textwidth}|}{\textbf{Temat projektu:} Zaprojektować i wdrożyć system informatyczny na potrzeby przedsiębiorstwa zgodnie z założeniami.}                                                                              \\ \hline
      \multicolumn{2}{|p{\textwidth}|}{\textbf{Założenia projektu:} AD DS, DNS, DHCP, GPO, WDS, RAID5, iSCSI, Serwer Wydruków (Print Server), Serwer WWW, FAILOVER CLUSTERING z rolami: SFS, DHCP Server, automatyzacja - skrypty PowerShell.} \\ \hline
      \begin{tabular}[c]{@{}l@{}}\textbf{Nazwisko i imię:} Maciej Wójs\\ \\ \textbf{Nr grupy:} L3\end{tabular}          &
      \textbf{Data oddania:} \today                                                                                                                                                                                                                   \\ \hline
      \begin{tabular}[c]{@{}l@{}}\textbf{Nazwa serwera SDC:} SDC06\\ \\ \textbf{Nazwa domeny AD:} BinaryBuilders.ad\end{tabular} &
      \textbf{Ocena:}                                                                                                                                                                                                                          \\ \hline
    \end{tabular}%
  }
\end{table}

\clearpage
 % tabela z danymi projektu

\pagestyle{fancy}
\newpage

%formatowanie spisu treści i~nagłówków
\renewcommand{\cftbeforesecskip}{8pt}
\renewcommand{\cftsecafterpnum}{\vskip 8pt}
\renewcommand{\cftparskip}{3pt}
\renewcommand{\cfttoctitlefont}{\Large\bfseries\sffamily}
\renewcommand{\cftsecfont}{\bfseries\sffamily}
\renewcommand{\cftsubsecfont}{\sffamily}
\renewcommand{\cftsubsubsecfont}{\sffamily}
\renewcommand{\cftparafont}{\sffamily}
%koniec formatowania spisu treści i nagłówków

\tableofcontents    %spis treści
\thispagestyle{fancy}
\newpage


\newpage

%%%%%%%%%%%%%%%%%%% treść główna dokumentu %%%%%%%%%%%%%%%%%%%%%%%%%

%! Pobieranie PDF
% ctrl + shift + ~
% komenda: upload.sh

%! SKRÓTY KLAWISZOWE
% LINK do skrótów klawiszowych: https://github.com/James-Yu/latex-workshop/wiki/Snippets
% ctrl+alt+j - przeniesienie z kodu do pdf
% ctrl + click - przeniesienie z pdf do kodu (dokument.pdf)
% zaznaczony fragment kodu -> ctrl+l -> ctrl+w
% gdy kuror na sekcji itp. -> cltr + alt + ] - obniżenie sekcji
% gdy kuror na sekcji itp. -> cltr + alt + [ - podniesienie sekcji
% kopia lini kodu -> ctrl + shift + strzałka w dół

% * SPOSOBY UZYWANIA MAKRA HERE * # 
% ? Listingi
% Parametr #1: Opis listingu (wyświetla sie bezpośrednio pod listingiem)
% Parametr #2 : Nazwa pliku oraz ID do oznaczania (wazne, zeby byl w katalogu kod oraz jego rozszerzenie to txt) 
% ? Zdjęcia
% OPCJONALNY Parametr #1: Szerokość zdjęcia (domyślnie jest to szerokość paragrafu ale jak sie poda w [#1] to wtedy zmienia się na podaną wartość)
% Parametr #2: Nazwa pliku z rozszerzeniem (podajemy katalog w którym jest plik i jego rozszerzenie np. rys/nazwa-rysunku.png)
% Parametr #3: Opis tego co jest na rysunku (wyświetla sie bezpośrednio pod rysunkiem) 
% Parametr #4: Identyfikator rysunku (do oznaczania zdjęć w tekście) 

\newpage
\section{Założenia projektowe – wymagania}		%1
\begin{enumerate}
    \item Utworzenie własnej domeny AD według formatu firma.ad, gdzie firma to nazwa firmy dla której
          przygotowujemy projekt.
    \item Autoryzacja pracownika przy użyciu imiennego konta działającego na wszystkich komputerach w
          sieci firmowej. Login zgodny ze schematem: imie.nazwisko. Utworzyć minimum po trzy konta dla
          każdego wydziału. Tworzenie grup oraz kont w domenie wykonywać przez skrypt, który odczyta
          dane grupy, konta z pliku i utworzy w domenie AD.
    \item Pracownicy powinni należeć do grupy globalnej odpowiedniej dla wydziału w którym
          pracują(przewidzieć 5 przykładowych wydziałów, np. kadry, place, gospodarczy, marketing, itp wg.
          własnego uznania). W przedsiębiorstwie przewidziano wydział informatyczny, którego pracownicy
          mają w pełni administrować domeną przedsiębiorstwa. Tworzenie grup oraz kont w domenie
          wykonywać przez skrypt, który odczyta dane grupy, konta z pliku i utworzy w domenie AD.
    \item Pracownicy powinni korzystać z zasobów sieciowych o nazwach: wspolny, oraz zasób wydziałowy
          (oddzielny zasób dla każdego wydziału). Zasoby powinny być udostępnione poprzez klaster pracy
          awaryjnej, który powinien korzystać z przestrzeni dyskowej (macierzy RAID-1) udostępnionej
          poprzez iSCSI o przestrzeni wypadkowej 30GB. Jeśli pozwalają zasoby sprzętowe to programową
          macierz RAID-1, oraz iSCSI Target Serwer można zainstalować na oddzielnym serwerze o nazwie
          SMPXX.firma.ad.
    \item Zasób wspolny ma być mapowany użytkownikowi jako dysk (patrz: założenia projektowej),
          natomiast zasób wydziałowy jako jeden z dysków (patrz: założenia projektowej) adekwatnie do
          grupy wydziałowej w której się pracownik znajduje. Mechanizm mapowania automatyczny przy
          użyciu polis GPO.
    \item System ma umożliwić instalację stacji klienckich z obrazu udostępnionego na serwerze.
    \item Pracownicy powinni mieć dostęp do drukarek sieciowych udostępnianych poprzez serwer wydruków. Serwer wydruków można zainstalować na oddzielnym serwerze, np. o nazwie \texttt{SPRXXfirma.ad}, lub w przypadku małych zasobów, na serwerze \texttt{SDC}. Dostęp do drukarki \texttt{\textbackslash\textbackslash SPRXX-firma.ad\textbackslash nazwa-drukarki}.
          \newpage
    \item Konfiguracja stacji klienckich w sposób automatyczny.
    \item Na klastrze pracy awaryjnej należy wdrożyć DHCP Serwer.
    \item Po udanym wdrożeniu serwera DHCP, wyłączyć serwer DHCP pracujący na SDC
    \item Wdrożyć serwer WWW wraz z firmową stroną (nie domyślną) dostępną pod adresem
          www.firma.ad.
    \item Wdrożyć WordPress współpracujący z usługą IIS Windows Serwera 2022.
\end{enumerate}
\subsection{Założenia projektowe}
\begin{itemize}
    \item nazwy serwerów powinny być zgodne ze schematem, przykładowo: SDCXX-NSACZ.firma.ad. - gdzie
          XX oznacza ostatnie dwie cyfry w numerze albumu,
    \item przyjąć adresację sieci jak niżej: 192.168.XX.40/24 - gdzie XX oznacza ostatnie dwie cyfry w numerze albumu,
    \item Zasoby sieciowe udostępnione na klastrze FC powinny być widoczne dla komputerów w postaci ścieżki UNC jako \texttt{\textbackslash\textbackslash sfsXX.firma.ad\textbackslash zasob}, \\ oraz \texttt{\textbackslash\textbackslash sfsXX.firma.ad\textbackslash wydzial} (nazwa wydziału), gdzie \texttt{XX} oznacza ostatnie dwie cyfry w numerze albumu.
    \item mapowanie dysków zgodnie ze schematem:
          zasób wspolny jako dysk X: , dyski wydziałowe(oddzielny zasób dla każdego wydziału) od G do K,
    \item  zaprojektować polisy tworzącą katalog c:\textbackslash ProjektyXX na stacjach końcowych, oraz ustawiającą
          zmienną środowiskową FIRMA=Nazwa, gdzie XX oznacza ostatnie dwie cyfry w numerze albumu,
    \item zakres adresacji IP dla stacji końcowych w zakresie od 100 ... do 200, uwzględnić funkcjonalność DHCP Serwer’a uruchomionego na klastrze,
    \item na stacji zainstalować drukarkę sieciową udostępnioną przez serwer wydruków i pokazać jej
          działanie.
\end{itemize}

\newpage
\section{Opis użytych technologii}		%2
%(W podpunktach dokonać krótkiej charakterystyki użytych technologii ) 


W niniejszym rozdziale przedstawiono szczegółową charakterystykę technologii wykorzystanych w ramach realizowanego projektu. Opis obejmuje funkcjonalność, zastosowania oraz znaczenie każdej z nich w kontekście budowy nowoczesnych środowisk IT.

\subsection{Active Directory Domain Services (AD DS)}
Active Directory Domain Services \footnote{Active Directory na Wikipedii\cite{ActiveDirectory}} (AD DS) to centralna usługa katalogowa firmy Microsoft, zaprojektowana w celu zarządzania zasobami sieciowymi w sposób efektywny i skalowalny. AD DS pozwala na tworzenie hierarchicznych struktur domenowych, które umożliwiają organizowanie użytkowników, komputerów, grup, drukarek i innych zasobów. Kluczowe funkcje AD DS obejmują:
- Centralne uwierzytelnianie użytkowników i urządzeń,
- Zarządzanie politykami bezpieczeństwa poprzez Group Policy,
- Możliwość delegowania uprawnień administracyjnych,
- Integrację z innymi usługami sieciowymi, takimi jak DNS czy DHCP.
Dzięki AD DS organizacje mogą efektywnie zarządzać dużymi środowiskami IT, zwiększając bezpieczeństwo i wydajność operacyjną.

\subsection{Domain Name System (DNS)}
Domain Name System\footnote{Więcej informacji o DNS\cite{dns}} (DNS) to podstawowy element infrastruktury sieciowej, który odpowiada za zamianę nazw domenowych (np. www.przyklad.com) na adresy IP wykorzystywane przez urządzenia sieciowe. DNS umożliwia łatwe lokalizowanie serwerów, aplikacji i innych zasobów. Najważniejsze cechy DNS to:
- Rozproszona struktura, która zapewnia skalowalność i niezawodność,
- Obsługa zapytań o różne rekordy, takie jak A (adres IPv4), AAAA (adres IPv6), MX (serwer poczty), czy CNAME (aliasy),
- Mechanizmy redundancji i buforowania, które zwiększają wydajność i odporność na awarie.
DNS jest fundamentem działania internetu i wielu aplikacji sieciowych, dlatego jego poprawna konfiguracja i zarządzanie mają kluczowe znaczenie.

\subsection{Dynamic Host Configuration Protocol (DHCP)}
Dynamic Host Configuration Protocol (DHCP) automatyzuje proces przypisywania adresów IP oraz innych parametrów sieciowych urządzeniom w sieci. Dzięki DHCP administratorzy mogą eliminować ręczną konfigurację każdego urządzenia, co jest szczególnie istotne w dużych środowiskach. Funkcje DHCP obejmują:
- Dynamiczne przydzielanie adresów IP z puli adresowej,
- Przekazywanie informacji o bramie domyślnej, serwerach DNS i czasie dzierżawy adresu,
- Obsługę rezerwacji adresów IP dla konkretnych urządzeń.
DHCP znacznie upraszcza zarządzanie sieciami, minimalizując ryzyko konfliktów adresów IP i błędów konfiguracyjnych.

\subsection{Group Policy Objects (GPO)}
Group Policy Objects (GPO) to mechanizm pozwalający administratorom na definiowanie i egzekwowanie ustawień konfiguracyjnych oraz polityk bezpieczeństwa w środowisku domenowym. Dzięki GPO możliwe jest centralne zarządzanie ustawieniami systemów operacyjnych, aplikacji i środowiska użytkownika. Przykładowe zastosowania GPO to:
- Automatyczne wdrażanie oprogramowania na komputerach użytkowników,
- Konfiguracja polityk haseł i zabezpieczeń,
- Ustawienia pulpitu, mapowanie dysków sieciowych oraz drukarek.
GPO zapewniają elastyczność i kontrolę w dużych środowiskach, ułatwiając przestrzeganie standardów korporacyjnych.

\subsection{Windows Deployment Services (WDS)}
Windows Deployment Services (WDS) to narzędzie, które umożliwia administratorom instalację systemów operacyjnych na komputerach poprzez sieć, bez potrzeby używania nośników fizycznych. WDS wspiera scenariusze takie jak:
- Wdrażanie systemów Windows w trybie bezobsługowym (unattended),
- Obsługa obrazów systemowych w formacie WIM i VHD,
- Konfiguracja obrazów rozruchowych i instalacyjnych.
WDS pozwala na efektywne wdrażanie systemów w środowiskach, w których konieczna jest szybka instalacja na wielu urządzeniach jednocześnie.

\subsection{RAID 1}
RAID 1 to technologia macierzy dyskowej znana również jako mirroring (lustrzanie), która zapewnia pełną redundancję danych. Dane są zapisywane jednocześnie na dwóch dyskach, co gwarantuje ich dostępność w przypadku awarii jednego z nich. Kluczowe cechy RAID 1 obejmują:
- Wysoką odporność na utratę danych dzięki pełnej kopii zapasowej,
- Łatwość odtwarzania danych w przypadku awarii,
- Lepszą wydajność odczytu dzięki możliwości równoczesnego odczytu z dwóch dysków.
RAID 1 jest idealnym rozwiązaniem dla aplikacji wymagających wysokiej niezawodności, takich jak serwery baz danych czy systemy krytyczne.

\subsection{iSCSI}
Internet Small Computer Systems Interface (iSCSI) umożliwia przesyłanie poleceń SCSI przez sieci IP, co pozwala na budowanie sieciowych systemów pamięci masowej (SAN). iSCSI zapewnia:
- Niskie koszty wdrożenia w porównaniu z tradycyjnymi technologiami SAN,
- Łatwość integracji z istniejącymi infrastrukturami sieciowymi,
- Wysoką wydajność dzięki zastosowaniu dedykowanych protokołów.
iSCSI jest idealnym rozwiązaniem dla firm poszukujących skalowalnych i przystępnych cenowo technologii przechowywania danych.

\subsection{Serwer Wydruków (Print Server)}
Serwer Wydruków centralizuje zarządzanie drukarkami i zadaniami drukowania w środowisku sieciowym. Jego funkcje obejmują:
- Udostępnianie drukarek w sieci lokalnej,
- Monitorowanie zadań drukowania i zużycia materiałów eksploatacyjnych,
- Zarządzanie uprawnieniami dostępu do drukarek.
Dzięki serwerowi wydruków administratorzy mogą zwiększyć wydajność i kontrolę nad procesami drukowania, jednocześnie obniżając koszty operacyjne.

\subsection{Serwer WWW}
Serwer WWW, taki jak Internet Information Services (IIS), obsługuje aplikacje webowe oraz strony internetowe, umożliwiając ich udostępnianie użytkownikom w sieci lokalnej i internecie. Kluczowe funkcje serwera WWW to:
- Obsługa protokołów HTTP, HTTPS, FTP i SMTP,
- Integracja z technologiami ASP.NET i PHP,
- Skalowalność i wsparcie dla aplikacji wymagających wysokiej wydajności.
Serwery WWW odgrywają istotną rolę w nowoczesnych infrastrukturach IT, umożliwiając realizację usług online.

\subsection{Failover Clustering z rolami SFS i DHCP Server}
Failover Clustering to technologia zapewniająca wysoką dostępność aplikacji i usług poprzez połączenie serwerów w klastry. Role takie jak Scale-Out File Server (SFS) i DHCP Server umożliwiają:
- Ciągłość działania usług w przypadku awarii jednego z węzłów klastra,
- Łatwą skalowalność w celu dostosowania do rosnących wymagań,
- Efektywne zarządzanie zasobami dyskowymi i sieciowymi.
Failover Clustering znajduje zastosowanie w krytycznych środowiskach, gdzie nieprzerwane działanie usług jest kluczowe.

\subsection{Automatyzacja - Skrypty PowerShell}
PowerShell to zaawansowane środowisko skryptowe i powłoka poleceń zaprojektowana z myślą o administracji systemami Windows. Funkcjonalności PowerShell obejmują:
- Automatyzację zadań, takich jak zarządzanie użytkownikami, konfiguracja systemów i monitorowanie zasobów,
- Obsługę zdalnych operacji za pomocą protokołu WinRM,
- Tworzenie modułów i skryptów dostosowanych do specyficznych potrzeb organizacji.
PowerShell jest narzędziem nieodzownym dla administratorów IT, umożliwiającym realizację nawet najbardziej złożonych operacji w sposób efektywny i powtarzalny.

	\newpage
\section{Schemat logiczny}		%3
% (Ma zawierać aktualne nazewnictwo i adresację IP.)
	\newpage
\section{Procedury instalacyjne poszczególnych usług}		%4
% Procedury instalacyjne poszczególnych usług.
% (W podpunktach zamieścić polecenia dotyczące instalacji wdrażanych usług) 

\subsection{Instalacja serwera}

Projekt rozpoczęto od instalacji systemu Windows Server 2022 na serwerze zainstalowanym w maszynie wirtualnej. Proces instalacji serwera jest widoczny na \OznaczZdjecie[rysunku]{sdc-install}. Po zainstalowaniu systemu, nadano serwerowi nazwę \texttt{SDC06} oraz przydzielono adres IP 192.168.6.40/24 co jest widoczne na \OznaczZdjecie[zdjęciu]{sdc-adres}.

\fg{rys/SDC-instalacja/1.png}{Instalacja serwera SDC}{sdc-install}
\clearpage
\fg{rys/SDC-instalacja/2.png}{Adresacja serwera SDC}{sdc-adres}
\subsection{Instalacja usługi Active Directory Domain Services (AD DS)}
Dalszym krokiem była instalacja usługi Active Directory Domain Services (AD DS) na serwerze \texttt{SDC06}.

\subsubsection{Proces instalacji AD DS}
\begin{enumerate}
\item Otwarcie \texttt{Server Manager}
\item Wybranie \texttt{Manage} $\rightarrow$ \texttt{Add Roles and Features}
\item Kliknięcie \texttt{Next}
\item Wybranie \texttt{Role-based or feature-based installation}. \OznaczZdjecie{ad-ds1}
\item Wybranie serwera \texttt{SDC06} i kliknięcie \texttt{Next}. \OznaczZdjecie{ad-ds2}
\item Wybranie roli \texttt{Active Directory Domain Services}.\OznaczZdjecie{ad-ds3} i \OznaczZdjecie[rysunek]{ad-ds4}
\item Kliknięcie \texttt{Next} na ekranie z funkcjami. \OznaczZdjecie{ad-ds5}
\item Kliknięcie \texttt{Next} na ekranie z informacjami dotyczącymi AD. \OznaczZdjecie{ad-ds6}
\item Kliknięcie \texttt{Next} na ekranie podsumowania. \OznaczZdjecie{ad-ds7}
\item Kliknięcie \texttt{Install}
\item Ponowne uruchomienie serwera
\end{enumerate}

\fg{rys/uslugi/AD/instalacja/1.png}{Instalacja oparta na rolach lub funkcjach}{ad-ds1}
\fg{rys/uslugi/AD/instalacja/2.png}{Ekran z wyborem SDC}{ad-ds2}
\clearpage
\fg{rys/uslugi/AD/instalacja/3.png}{Dodanie usługi AD DS}{ad-ds3}
\fg{rys/uslugi/AD/instalacja/4.png}{Wybór roli serwera: AD DS}{ad-ds4}
\clearpage
\fg{rys/uslugi/AD/instalacja/5.png}{Wybór funcji serwera}{ad-ds5}
\fg{rys/uslugi/AD/instalacja/6.png}{Ekran z informacjami dotyczącymi AD}{ad-ds6}
\clearpage
\fg{rys/uslugi/AD/instalacja/7.png}{ekran podsumowania instalacji AD}{ad-ds7}
\fg{rys/uslugi/AD/instalacja/8.png}{Progress instalacji}{ad-ds8}
\clearpage
\fg{rys/uslugi/AD/instalacja/9.png}{Zainstalowane AD}{ad-ds9}


\subsubsection{Konfiguracja usługi AD DS}
Dalszym krokiem było utworzenie domeny \texttt{BinaryBuilders.ad} na serwerze oraz skonfigurowanie lasu w usłudze Active Directory Domain Services (AD DS). Proces ten obejmował kilka kluczowych etapów:
\begin{enumerate}
	\item Podniesienie poziomu serwera do kontrolera domeny. \OznaczZdjecie{ad-ds10}
	\item Stworzenie lasu. \OznaczZdjecie{ad-ds11}
	\item Wybór hasła dla trybu DSRM. \OznaczZdjecie{ad-ds12}
	\item Konfiguracja DNS. \OznaczZdjecie{ad-ds13}
	\item Dodanie nazwy NetBIOS \texttt{BINARYBUILDERS}. \OznaczZdjecie{ad-ds14}
	\item Wybór ścieżki do plików. \OznaczZdjecie{ad-ds15}
	\item Przegląd ustawień i wyświetlenie skryptu instalacji. \OznaczZdjecie{ad-ds16}
	\item Sprawdzenie wymagań komputera do instalacji AD. \OznaczZdjecie{ad-ds17}
	\item Poprawne sprawdzenie wymagań komputera do instalacji AD. \OznaczZdjecie{ad-ds18}
	\item Rozpoczęcie instalacji. \OznaczZdjecie{ad-ds19}
	\item Ekran informujący o wylogowaniu. \OznaczZdjecie{ad-ds20}

\end{enumerate}

\fg{rys/uslugi/AD/konfiguracja/1.png}{Podniesienie poziomu serwera do kontrolera domeny}{ad-ds10}
\fg{rys/uslugi/AD/konfiguracja/2.png}{Stworzenie lasu. Domena BinaryBuilders.ad}{ad-ds11}
\clearpage
\fg{rys/uslugi/AD/konfiguracja/3.png}{Wybór hasła dla trybu DSRM (domyślne opcje)}{ad-ds12}

\fg{rys/uslugi/AD/konfiguracja/4.png}{Konfiguracja DNS}{ad-ds13}
\clearpage
\fg{rys/uslugi/AD/konfiguracja/5.png}{Dodanie nazwy NetBIOS BINARYBUILDERS}{ad-ds14}
\fg{rys/uslugi/AD/konfiguracja/6.png}{Wybór ścieżki do plików (Domyślne opcje)}{ad-ds15}
\clearpage
\fg{rys/uslugi/AD/konfiguracja/7.png}{Przegląd ustawień i wyświetlenie skryptu instalacji}{ad-ds16}
\fg{rys/uslugi/AD/konfiguracja/8.png}{Sprawdzenie wymagań komputera do instalacji AD}{ad-ds17}
\clearpage
\fg{rys/uslugi/AD/konfiguracja/9.png}{Poprawne sprawdzenie wymagań}{ad-ds18}
\fg{rys/uslugi/AD/konfiguracja/10.png}{Rozpoczęcie instalacji}{ad-ds19}
\clearpage
\fg{rys/uslugi/AD/konfiguracja/11.png}{Ekran informujący o wylogowaniu w krótce}{ad-ds20}
\fg{rys/uslugi/AD/konfiguracja/12.png}{Zalogowanie się na konto administratora domeny}{ad-ds22}
\clearpage


\subsection{Instalacja usługi Domain Name System (DNS)}
Po zainstalowaniu usługi Active Directory Domain Services (AD DS) przystąpiono do instalacji usługi Domain Name System (DNS) na serwerze \texttt{SDC06}, ponieważ jest to niezbędne do prawidłowego funkcjonowania AD DS, a ja zapomniałem wykonać tego kroku wcześniej.
\subsubsection{Proces instalacji DNS}
\begin{enumerate}
	\item Otwarcie \texttt{Server Manager}
	\item Wybranie \texttt{Manage} $\rightarrow$ \texttt{Add Roles and Features}
	\item Kliknięcie \texttt{Next}
	\item Wybranie \texttt{Role-based or feature-based installation}. \OznaczZdjecie{dns1}
	\item Wybranie serwera \texttt{SDC06} i kliknięcie \texttt{Next}. \OznaczZdjecie{dns2}
	\item Wybranie roli \texttt{DNS Server}.\OznaczZdjecie{dns3} i \OznaczZdjecie[rysunek]{dns4}
	\item Kliknięcie \texttt{Next} na ekranie z funkcjami. \OznaczZdjecie{dns5}
	\item Kliknięcie \texttt{Next} na ekranie z informacjami dotyczącymi DNS.
	\item Kliknięcie \texttt{Next} na ekranie podsumowania. \OznaczZdjecie{dns6}
	\item Kliknięcie \texttt{Install} na ekranie z postępem instalacji. \OznaczZdjecie{dns7}
\end{enumerate}

\fg{rys/uslugi/DNS/instalacja/1.png}{Instalacja oparta na rolach lub funkcjach}{dns1}
\fg{rys/uslugi/DNS/instalacja/2.png}{Ekran z wyborem SDC}{dns2}
\clearpage
\fg{rys/uslugi/DNS/instalacja/3.png}{Dodanie usługi DNS}{dns3}
\fg{rys/uslugi/DNS/instalacja/4.png}{Wybór roli serwera: DNS}{dns4}
\clearpage

\fg{rys/uslugi/DNS/instalacja/5.png}{Wybór funcji serwera}{dns5}
\fg{rys/uslugi/DNS/instalacja/6.png}{Ekran podsumowania}{dns6}
\clearpage
\fg{rys/uslugi/DNS/instalacja/7.png}{Postęp instalacji DNS}{dns7}


\subsubsection{Proces konfiguracji DNS}
Po zainstalowaniu usługi DNS przystąpiono do konfiguracji usługi. Proces ten obejmował kilka kluczowych etapów:

\begin{enumerate}
	\item Otwarcie \texttt{DNS Manager} z poziomu \texttt{Server Manager} \OznaczZdjecie{dns8}
	\item Wybranie \texttt{SDC06.BinaryBuilders.ad} \OznaczZdjecie{dns9}
	\item Dodanie nowego hosta typu \texttt{A} o nazwie \texttt{www} i adresie IP serwera \texttt{SDC06}: 192.168.6.40 w strefie przeszukiwania do przodu. \OznaczZdjecie{dns10}
	\item Dodanie strefy przeszukiwania do tyłu. \OznaczZdjecie{dns11}
	\begin {enumerate}
	\item Kliknięcie \texttt{Next w oknie dodawania strefy przeszukiwania do tyłu} \OznaczZdjecie{dns12}
	\item Wybranie strefy podstawowej \texttt{BinaryBuilders.ad} \OznaczZdjecie{dns12}
	\item Wybranie strefy replikacji \texttt{To all DNS servers running on domain controllers in this domain: BinaryBuilders.ad} \OznaczZdjecie{dns13}
	\item Wybór IPv4 \OznaczZdjecie{dns14}
	\item Wybór nazwy lub adresu IP translacji \OznaczZdjecie{dns15}
	\item Wybór dynamicznego odświeżania strefy \OznaczZdjecie{dns16}
	\item Kliknięcie \texttt{Finish} \OznaczZdjecie{dns17}
	\end{enumerate}
	\item Dodanie wskaźnika ptr do strefy przeszukiwania do tyłu. \OznaczZdjecie{dns18}
\end{enumerate}

\fg{rys/uslugi/DNS/konfiguracja/1.png}{Otwarcie DNS Manager}{dns8}
\clearpage
\fg{rys/uslugi/DNS/konfiguracja/2.png}{Wybranie SDC06.BinaryBuilders.ad}{dns9}
\clearpage
\fg{rys/uslugi/DNS/konfiguracja/3.png}{Dodanie nowego hosta typu A}{dns10}
\clearpage
\fg{rys/uslugi/DNS/konfiguracja/4.png}{Dodanie strefy przeszukiwania do tyłu}{dns11}
\clearpage
\fg{rys/uslugi/DNS/konfiguracja/5.png}{Okno dodawania strefy}{dns12}
\clearpage
\fg{rys/uslugi/DNS/konfiguracja/6.png}{Wybór strefy podstawowej}{dns12}
\clearpage
\fg{rys/uslugi/DNS/konfiguracja/7.png}{Wybór strefy replikacji}{dns13}
\clearpage
\fg{rys/uslugi/DNS/konfiguracja/8.png}{Wybór IPv4}{dns14}
\clearpage
\fg{rys/uslugi/DNS/konfiguracja/9.png}{Wybór nazwy lub adresu IP translacji}{dns15}
\clearpage
\fg{rys/uslugi/DNS/konfiguracja/10.png}{Wybór dynamicznego odświeżania strefy}{dns16}
\clearpage
\fg{rys/uslugi/DNS/konfiguracja/11.png}{Zakończenie dodawania strefy}{dns17}
\clearpage
\fg{rys/uslugi/DNS/konfiguracja/12.png}{Dodanie wskaźnika ptr}{dns18}
\clearpage

\subsection{Skrypt dodający użytkowników i grupy do domeny}
Skrypt służy do automatycznego tworzenia użytkowników i grup w Active Directory na podstawie danych z pliku CSV. Skrypt jest napisany w języku PowerShell i wykorzystuje moduł Active Directory.
\subsubsection{Wytłumaczenie działania skryptu}

\paragraph{Główne komponenty}

\begin{verbatim}
$groupsOU = "OU=Grupy,$ou"
\end{verbatim}
Definiuje ścieżkę do kontenera grup w strukturze AD.

Skrypt iteruje przez każdy wiersz pliku CSV:
\begin{itemize}
    \item Tworzy nazwę użytkownika (imię.nazwisko)
    \item Normalizuje adres email
    \item Przypisuje stanowisko do grupy
\end{itemize}

\paragraph{Zarządzanie grupami}
\begin{itemize}
    \item Sprawdza istnienie grupy
    \item Tworzy nową grupę jeśli nie istnieje
    \item Zakres grupy: Global
    \item Lokalizacja: OU=Grupy
\end{itemize}

\paragraph{Zarządzanie użytkownikami}
Przy wykryciu istniejącej nazwy użytkownika:
\begin{itemize}
    \item Dodaje losowy numer (1–999)
    \item Aktualizuje adres email
    \item Sprawdza ponownie unikalność
\end{itemize}

Parametry nowego konta:
\begin{itemize}
    \item SamAccountName
    \item UserPrincipalName
    \item Pełna nazwa
    \item Email służbowy
    \item Dział
    \item Hasło (konwertowane na SecureString)
\end{itemize}

\paragraph{Ustawienia bezpieczeństwa}
\begin{itemize}
    \item Konto aktywne od razu
    \item Hasło nigdy nie wygasa
    \item Brak wymuszonej zmiany hasła
\end{itemize}


\paragraph{Wnioski}
Skrypt jest efektywnym narzędziem do automatyzacji procesu tworzenia kont użytkowników i grup w Active Directory. Może być łatwo dostosowany do różnych scenariuszy i wymagań.

\subsubsection{Skrypt}
\ListingFile{CSV}{CSV}
\clearpage
\subsubsection{Przykładowe dane do skryptu}
\ListingFile{Dane}{Dane}
\clearpage

\subsection{Instalacja usługi RAID 1}
Następnie przystąpiono do instalacji usługi RAID 1 na serwerze \texttt{SDC06}. Aby to zrobić, należy wykonać następujące kroki:
\begin{enumerate}
	\item Dodanie dwóch dysków o poejmnośći 30GB do maszyny wirtualnej w programie VirtualBox\footnote{Strona projektu VirtualBox\cite{VirtualBox}}. \OznaczZdjecie{raid1-1} %chktex 8
	\item Otwarcie \texttt{Zarządzania dyskami} i inicjalizacja dysków \OznaczZdjecie{raid1-2} %chktex 8
	\item Wynik inicjalizacji dysków. \OznaczZdjecie{raid1-3} %chktex 8
	\item Wybranie \texttt{New Mirrored Volume} jako typu RAID. \OznaczZdjecie{raid1-4} %chktex 8
	\item Dodanie drugiego dysku do stworzenia macierzy. \OznaczZdjecie{raid1-5} %chktex 8
	\item Wynik dodania drugiego dysku. \OznaczZdjecie{raid1-6} %chktex 8
	\item Brak przypisanej litery do macierzy RAID 1. \OznaczZdjecie{raid1-7} %chktex 8
	\item Wybranie braku formatowania. \OznaczZdjecie{raid1-8} %chktex 8
	\item Ekran potwierdzenia. \OznaczZdjecie{raid1-9} %chktex 8
	\item Wynik stworzenia macierzy. \OznaczZdjecie{raid1-10} %chktex 8
	\item Okno dodania litery dla RAID 1. \OznaczZdjecie{raid1-11} %chktex 8
	\item Ustawienie litery na \texttt{R:}. \OznaczZdjecie{raid1-12} %chktex 8
	\item Zakończenie konfigurowania macierzy – wynik. \OznaczZdjecie{raid1-13} %chktex 8
	\item Nadanie etykiety RAID 1. \OznaczZdjecie{raid1-14} %chktex 8
\end{enumerate}

\fg{rys/uslugi/RAID1/1.png}{Dodanie dysku do maszyny wirtualnej}{raid1-1} %chktex 8
\fg{rys/uslugi/RAID1/2.png}{Otwarcie Zarządzania dyskami i inicjalizacja dysków}{raid1-2} %chktex 8
\clearpage
\fg{rys/uslugi/RAID1/3.png}{Wynik inicjalizacji dysków}{raid1-3} %chktex 8
\fg{rys/uslugi/RAID1/4.png}{Wybranie New Mirrored Volume}{raid1-4} %chktex 8
\clearpage
\fg{rys/uslugi/RAID1/5.png}{Dodanie drugiego dysku w konfiguratorze}{raid1-5} %chktex 8
\fg{rys/uslugi/RAID1/6.png}{Wynik dodania drugiego dysku w konfuguratorze}{raid1-6} %chktex 8
\clearpage
\fg{rys/uslugi/RAID1/7.png}{Brak przypisania litery do macierzy Raid1}{raid1-7} %chktex 8
\fg{rys/uslugi/RAID1/8.png}{Wybranie braku formatowania}{raid1-8} %chktex 8
\clearpage
\fg{rys/uslugi/RAID1/9.png}{Ekran potwierdzenia}{raid1-9} %chktex 8
\fg{rys/uslugi/RAID1/10.png}{Wynik stworzenia macierzy}{raid1-10} %chktex 8
\clearpage
\fg{rys/uslugi/RAID1/11.png}{Okno dodania litery dla raid1}{raid1-11} %chktex 8
\clearpage
\fg{rys/uslugi/RAID1/12.png}{Ustawienie litery na R:}{raid1-12} %chktex 8
\fg{rys/uslugi/RAID1/13.png}{Wynik ustawienia litery}{raid1-13} %chktex 8
\clearpage
\fg{rys/uslugi/RAID1/14.png}{Nadanie etykiety RAID1}{raid1-14} %chktex 8
\clearpage

\subsection{Instalacja usługi iSCSI target server}
Instalacja usługi iSCSI umożliwia konfigurację urządzeń pamięci masowej przez sieć, co jest szczególnie przydatne w środowiskach wirtualnych i serwerowych. Poniżej przedstawiono kroki instalacji usługi iSCSI na serwerze \texttt{SDC06}:
\begin{enumerate}
	\item Otwarcie \texttt{Server Manager}
	\item Wybranie \texttt{Manage} $\rightarrow$ \texttt{Add Roles and Features}
	\item Kliknięcie \texttt{Next}
	\item Wybranie \texttt{Role-based or feature-based installation}. \OznaczZdjecie{iscsi1}
	\item Wybranie serwera \texttt{SDC06} i kliknięcie \texttt{Next}. \OznaczZdjecie{iscsi2}
	\item Wybranie roli \texttt{iSCSI Target Server}.\OznaczZdjecie{iscsi3} i \OznaczZdjecie[rysunek]{iscsi4}
	\item Kliknięcie \texttt{Next} na ekranie z funkcjami. \OznaczZdjecie{iscsi5}
	\item Kliknięcie \texttt{Next} na ekranie z informacjami dotyczącymi iSCSI.
	\item Kliknięcie \texttt{Next} na ekranie podsumowania. \OznaczZdjecie{iscsi6}
	\item Kliknięcie \texttt{Install} na ekranie z postępem instalacji. \OznaczZdjecie{iscsi7}
	\item Otwarcie \texttt{iSCSI Target Server} z poziomu \texttt{Server Manager} \OznaczZdjecie{iscsi8}
	\item Kliknięcie \texttt{New iSCSI Virtual Disk} \OznaczZdjecie{iscsi9}
	\item Wybranie dysku do konfiguracji $\rightarrow$ \texttt{R:} \OznaczZdjecie{iscsi10}
	\item Wybranie nazwy dysku $\rightarrow$ \texttt{Dysk1} \OznaczZdjecie{iscsi11}
	\item Wybranie rozmiaru dysku $\rightarrow$ 30GB \OznaczZdjecie{iscsi12}
	\item Wybranie nazwy celu $\rightarrow$ \texttt{dysk1-target} \OznaczZdjecie{iscsi13}
	\item Wybranie serwera dostępu $\rightarrow$ \texttt{stacja kliencka} \OznaczZdjecie{iscsi14}
	\item Kliknięcie \texttt{Next} na ekranie z funkcjami. \OznaczZdjecie{iscsi15}
	\item Kliknięcie \texttt{Next} na ekranie z informacjami dotyczącymi iSCSI.
	\item 
\end{enumerate}

\fg{rys/uslugi/iSCSI/1.png}{Instalacja oparta na rolach lub funkcjach}{iscsi1}
\fg{rys/uslugi/iSCSI/2.png}{Ekran z wyborem SDC}{iscsi2}
\clearpage
\fg{rys/uslugi/iSCSI/3.png}{Dodanie usługi iSCSI}{iscsi3}
\fg{rys/uslugi/iSCSI/4.png}{Ekran podsumowania}{iscsi4}
\clearpage
\fg{rys/uslugi/iSCSI/5.png}{Otworzenie New iSCSI Virtual Disk}{iscsi5}
\fg{rys/uslugi/iSCSI/6.png}{Wybór dysku do konfiguracji $\rightarrow$ R:}{iscsi6}
\clearpage
\fg{rys/uslugi/iSCSI/7.png}{Wybór nazwy dysku $\rightarrow$ Dysk1}{iscsi7}
\fg{rys/uslugi/iSCSI/8.png}{Wybór rozmiaru dysku $\rightarrow$ 30GB}{iscsi8}
\clearpage
\fg{rys/uslugi/iSCSI/9.png}{Wybór nazwy celu $\rightarrow$ dysk1-target}{iscsi9}
W tym etapie wystąpił błąd, a następnie w dalszej części projektu dysk oraz cel zostały usunięte, ponieważ nie były częścią pierwotnych założeń projektu.
\fg{rys/uslugi/iSCSI/10.png}{Wybór serwera dostępu $\rightarrow$ stacja kliencka}{iscsi10}
\clearpage
\fg{rys/uslugi/iSCSI/11.png}{Ekran potwierdzenia ustawień iSCSI}{iscsi11}
\fg{rys/uslugi/iSCSI/12.png}{Wynik konfiguracji iSCSI}{iscsi12}
\clearpage
\fg{rys/uslugi/iSCSI/13.png}{Kliknięcie na właściwości celu z poziomu Server Manager}{iscsi13}
\fg{rys/uslugi/iSCSI/14.png}{Brak stacji klieckiej}{iscsi14}
\clearpage
\fg{rys/uslugi/iSCSI/15.png}{Otworzenie iSCSI Inicjator na stacji klieckiej}{iscsi15}
\fg{rys/uslugi/iSCSI/16.png}{Dodanie adresu ip serwera SDC06 w zakładce Discover Portal}{iscsi16}
\clearpage
\fg{rys/uslugi/iSCSI/17.png}{Powrót na SDC06 w celu dodania stacji klienckiej jako inicjatora}{iscsi17}
\fg{rys/uslugi/iSCSI/18.png}{Powrót na komputer użytkownika w celu odświeżania wykrywanych urządzeń}{iscsi18}
\clearpage
\fg{rys/uslugi/iSCSI/19.png}{Połączenie z wykrytym dyskiem}{iscsi19}
\fg{rys/uslugi/iSCSI/20.png}{Automatyczna konfiguracja dysku w zakładce Woluminy i urządzenia}{iscsi20}
\clearpage
\fg{rys/uslugi/iSCSI/21.png}{Inicjalizacja dysku}{iscsi21}
\fg{rys/uslugi/iSCSI/22.png}{Wybór typu partycji}{iscsi22}
\clearpage
\fg{rys/uslugi/iSCSI/23.png}{Wynik działania dysku}{iscsi23}

\subsection{Instalacja klastra failover}
Na serwerze \texttt{SDC06} za pomocą narzędzia \texttt{Server Manager} zainstalowano rolę Failover Clustering.

\fg{rys/Klaster/1.png}{Usunięcie dysku iSCSI dysk1}{Klaster1}
\fg{rys/Klaster/2.png}{Wybór dysku do konfiguracji $\rightarrow$ R: w New iSCSI Virtual Disk}{Klaster2}
\clearpage
\fg{rys/Klaster/3.png}{Wybór nazwy dysku $\rightarrow$ klaster}{Klaster3}
\fg{rys/Klaster/4.png}{Wybór rozmiaru dysku $\rightarrow$ 30GB}{Klaster4}
\clearpage
\fg{rys/Klaster/5.png}{Nowy cel}{Klaster5}
\fg{rys/Klaster/6.png}{Nazwa celu klaster-SN1-SN2}{Klaster6}
\clearpage
\fg{rys/Klaster/7.png}{Dodanie serwerów dostępu SN1 i SN2}{Klaster7}
\fg{rys/Klaster/8.png}{Ekran Enable Authentication – domyślne ustawienia}{Klaster8}
\clearpage
\fg{rys/Klaster/9.png}{Ekran potwierdzenia ustawień iSCSI}{Klaster9}
\fg{rys/Klaster/10.png}{Ekran z wynikami}{Klaster10}
\clearpage
\fg{rys/Klaster/11.png}{Połączenie z dyskiem na SN1 i SN2}{Klaster11}
\fg{rys/Klaster/12.png}{Automatyczna konfiguracja dysku na SN1 i SN2}{Klaster12}
\clearpage
\fg{rys/Klaster/13.png}{Dodanie serwerów z poziomu SDC06 dla łatwiejszego zarządzania}{Klaster13}
\fg{rys/Klaster/14.png}{Wyszukanie serwerów}{Klaster14}
\clearpage
\fg{rys/Klaster/15.png}{Dodanie SN1 i SN2}{Klaster15}
\fg{rys/Klaster/16.png}{Add roles And Features (SN1--06) z poziomu SDC06}{Klaster16}
\clearpage
\fg{rys/Klaster/17.png}{Instalacja oparta na rolach lub funkcjach}{Klaster17}
\fg{rys/Klaster/18.png}{Wybór serwerów}{Klaster18}
\clearpage
\fg{rys/Klaster/19.png}{Wybranie funcji Failover}{Klaster21}
\fg{rys/Klaster/20.png}{Rozpoczęcie instalacji}{Klaster22}
\clearpage
\fg{rys/Klaster/21.png}{Zakończenie instalacji}{Klaster19}
\fg{rys/Klaster/22.png}{Przestawienie dysku na online w SN1}{Klaster23}
\clearpage
\fg{rys/Klaster/23.png}{inicjalizacja dysku}{Klaster24}
\fg{rys/Klaster/24.png}{Formanie i nadanie etykiety dysku}{Klaster20}
\clearpage
\fg{rys/Klaster/25.png}{Sprawdzenie działania dysku na SN1 i SN2}{Klaster25}
\fg{rys/Klaster/26.png}{Otworzenie kreatora klastra i wybranie SN1 i SN2}{Klaster26}
\clearpage
\fg{rys/Klaster/27.png}{Nazwa klastra i nadanie IP}{Klaster27}
\fg{rys/Klaster/28.png}{Ekran podsumowania}{Klaster28}
\clearpage
\fg{rys/Klaster/29.png}{Ekran przetwarzania klastra}{Klaster29}
\fg{rys/Klaster/30.png}{Ekran podsumowania}{Klaster30}
\clearpage
\fg{rys/Klaster/31.png}{Sprawdzenie działania klastra}{Klaster31}

\subsection{SFS – instalacja i konfiguracja}
Na serwerze \texttt{SDC06} zainstalowano usługę SFS (Scale-Out File Server) w celu udostępnienia zasobów sieciowych dla klastra failover.

\fg{rys/uslugi/SFS/1.png}{Instalacja oparta na rolach lub funkcjach}{SFS1}
\fg{rys/uslugi/SFS/2.png}{Wybór serwerów}{SFS2}
\clearpage
\fg{rys/uslugi/SFS/3.png}{Wybranie funkcji File Server}{SFS3}
\fg{rys/uslugi/SFS/4.png}{Rozpoczęcie instalacji}{SFS4}
\clearpage
\fg{rys/uslugi/SFS/5.png}{Zakończenie instalacji}{SFS5}
\fg{rys/uslugi/SFS/6.png}{Wybór dysku do konfiguracji $\rightarrow$ R: w New iSCSI Virtual Disk}{SFS6}
\clearpage
\fg{rys/uslugi/SFS/7.png}{Wybór nazwy dysku $\rightarrow$ przestrzenFS}{SFS7}
\fg{rys/uslugi/SFS/8.png}{Wybór rozmiaru dysku $\rightarrow$ 10GB}{SFS8}
\clearpage
\fg{rys/uslugi/SFS/9.png}{Nowy cel}{SFS9}
\fg{rys/uslugi/SFS/10.png}{Nazwa celu dyskFS}{SFS10}
\clearpage
\fg{rys/uslugi/SFS/11.png}{Dodanie serwerów dostępu SN1 i SN2}{SFS11}
\fg{rys/uslugi/SFS/12.png}{Ekran potwierdzenia ustawień iSCSI}{SFS12}
\clearpage
\fg{rys/uslugi/SFS/13.png}{Ekran z wynikami}{SFS13}
\fg{rys/uslugi/SFS/14.png}{Pierwszy ekran kreatora roli klastra}{SFS14}
\clearpage
\fg{rys/uslugi/SFS/15.png}{Wybór roli file server}{SFS15}
\fg{rys/uslugi/SFS/16.png}{Wybranie typu FS}{SFS16}
\clearpage
\fg{rys/uslugi/SFS/17.png}{Nazwa serwera i adresu IP}{SFS17}
\fg{rys/uslugi/SFS/18.png}{Przestawienie dysku na online w SN1}{SFS18}
\clearpage
\fg{rys/uslugi/SFS/19.png}{Inicjalizacja dysku}{SFS19}
\fg{rys/uslugi/SFS/20.png}{Formatowanie i nadanie etykiety dysku}{SFS20}
\clearpage
\fg{rys/uslugi/SFS/21.png}{Dodanie dysku do klastra}{SFS21}
\fg{rys/uslugi/SFS/22.png}{Wynik dodania dysku do klastra}{SFS22}
\clearpage
\fg{rys/uslugi/SFS/23.png}{Kontynuacja dodawania roli FS dodanie dysku klastra do roli}{SFS23}
\fg{rys/uslugi/SFS/24.png}{Ekran potwierdzenia}{SFS24}
\clearpage
\fg{rys/uslugi/SFS/25.png}{Podsumowanie dodania roli FS}{SFS25}
\fg{rys/uslugi/SFS/26.png}{Add file share z poziomu Menadżera klastra}{SFS26}
\clearpage
\fg{rys/uslugi/SFS/27.png}{Wybór typu udziału}{SFS27}
\fg{rys/uslugi/SFS/28.png}{Wybór dysku do udziału}{SFS28}
\clearpage
\fg{rys/uslugi/SFS/29.png}{Wybór nazwy udziału – Wspolny}{SFS29}
\fg{rys/uslugi/SFS/30.png}{Wybór ustawień udziału}{SFS30}
\clearpage
\fg{rys/uslugi/SFS/31.png}{Ekran premisji}{SFS31}
\fg{rys/uslugi/SFS/32.png}{Ekran potwierdzenia}{SFS32}
\clearpage
\fg{rys/uslugi/SFS/33.png}{Wynik dodania udziału}{SFS33}
\fg{rys/uslugi/SFS/34.png}{Sprawdzenie działania udziału – cz. 1}{SFS34}
\clearpage
\fg{rys/uslugi/SFS/35.png}{Sprawdzenie działania udziału – cz. 2}{SFS35}
\fg{rys/uslugi/SFS/36.png}{Sprawdzenie działania udziału – cz. 3}{SFS36}
\clearpage
\fg{rys/uslugi/SFS/37.png}{Wybór nazwy udziału – Kadry}{SFS37}
\fg{rys/uslugi/SFS/38.png}{Wybór ustawień udziału}{SFS38}
\clearpage
\fg{rys/uslugi/SFS/39.png}{Usunięcie premisji}{SFS39}
\fg{rys/uslugi/SFS/40.png}{Nowe premisje}{SFS40}
\clearpage
\fg{rys/uslugi/SFS/41.png}{Wynik dodania udziału}{SFS41}
\fg{rys/uslugi/SFS/42.png}{Sprawdzenie działania udziału – Kadry}{SFS42}
\clearpage
\fg{rys/uslugi/SFS/43.png}{Wybór nazwy udziału – Ksiegowosc}{SFS43}
\fg{rys/uslugi/SFS/44.png}{Konwersja dziedziczonych premisji na bezpośrednie}{SFS44}
\clearpage
\fg{rys/uslugi/SFS/45.png}{Usunięcie dziedziczonych premisji}{SFS45}
\fg{rys/uslugi/SFS/46.png}{Nowe premisje}{SFS46}
\clearpage
\fg{rys/uslugi/SFS/47.png}{Sprawdzenie działania udziału – Księgowość}{SFS47}
\fg{rys/uslugi/SFS/48.png}{Wybór nazwy udziału – Marketing}{SFS48}
\clearpage
\fg{rys/uslugi/SFS/49.png}{Nowe premisje}{SFS49}
\fg{rys/uslugi/SFS/50.png}{Wynik dodania udziału}{SFS50}
\clearpage
\fg{rys/uslugi/SFS/51.png}{Wybór nazwy udziału – Programista}{SFS51}
\fg{rys/uslugi/SFS/52.png}{Nowe premisje}{SFS52}
\clearpage
\fg{rys/uslugi/SFS/53.png}{Wynik dodania udziału}{SFS53}
\fg{rys/uslugi/SFS/54.png}{Wybór nazwy udziału – IT}{SFS54}
\clearpage
\fg{rys/uslugi/SFS/55.png}{Nowe premisje}{SFS55}
\fg{rys/uslugi/SFS/56.png}{Wynik dodania udziału}{SFS56}

\subsection{Group Policy Objects (GPO)}
Na serwerze \texttt{SDC06} skonfigurowano polityki Group Policy Objects (GPO) w celu centralnego zarządzania ustawieniami komputerów oraz użytkowników w organizacji. Dzięki temu administratorzy mogą automatycznie wdrażać zasady bezpieczeństwa, konfigurację systemów operacyjnych oraz ustawienia aplikacji w sieci, eliminując potrzebę ręcznego konfigurowania każdego urządzenia. GPO zapewnia jednolitość w zarządzaniu urządzeniami,
\subsubsection{Polisy dotyczące dysków}
\fg{rys/uslugi/GPO/1.png}{Otwarcie Group Policy Management}{GPO1}
\fg{rys/uslugi/GPO/2.png}{Utworzenie nowej polityki}{GPO2}
\clearpage
\fg{rys/uslugi/GPO/3.png}{Wybranie Drive Maps $\rightarrow$ New $\rightarrow$ Mapped Drive}{GPO3}
\fg{rys/uslugi/GPO/4.png}{Konfiguracja dysku – Wspolny}{GPO4}
\clearpage
\fg{rys/uslugi/GPO/5.png}{Test dysku – Wspolny}{GPO5}
\fg{rys/uslugi/GPO/6.png}{Edycja dysku dla programisty}{GPO6}
\clearpage
\fg{rys/uslugi/GPO/7.png}{Wybranie Drive Maps $\rightarrow$ New $\rightarrow$ Mapped Drive}{GPO7}
\fg{rys/uslugi/GPO/8.png}{Konfiguracja dysku – cz. 1}{GPO8}
\clearpage
\fg{rys/uslugi/GPO/9.png}{Konfiguracja dysku – cz. 2}{GPO9}
\fg{rys/uslugi/GPO/10.png}{Konfiguracja dysku – cz. 3}{GPO10}
\clearpage
\fg{rys/uslugi/GPO/11.png}{Konfiguracja dysku – cz. 4}{GPO11}
\fg{rys/uslugi/GPO/12.png}{Konfiguracja dysku – cz. 5}{GPO12}
\clearpage
\fg{rys/uslugi/GPO/13.png}{Test dysku}{GPO13}
\fg{rys/uslugi/GPO/14.png}{Konfiguracja dysku dla kadr – cz. 1}{GPO14}
\clearpage
\fg{rys/uslugi/GPO/15.png}{Konfiguracja dysku dla kadr – cz. 2}{GPO15}
\fg{rys/uslugi/GPO/16.png}{Konfiguracja dysku dla kadr – cz. 3}{GPO16}
\clearpage
\fg{rys/uslugi/GPO/17.png}{Konfiguracja dysku dla ksiegowosc – cz. 1}{GPO17}
\fg{rys/uslugi/GPO/18.png}{Konfiguracja dysku dla ksiegowosc – cz. 2}{GPO18}
\clearpage
\fg{rys/uslugi/GPO/19.png}{Konfiguracja dysku dla ksiegowosc – cz. 3}{GPO19}
\fg{rys/uslugi/GPO/20.png}{Konfiguracja dysku dla Marketing – cz. 1}{GPO20}
\clearpage
\fg{rys/uslugi/GPO/21.png}{Konfiguracja dysku dla Marketing – cz. 2}{GPO21}
\fg{rys/uslugi/GPO/22.png}{Konfiguracja dysku dla Marketing – cz. 3}{GPO22}
\clearpage
\fg{rys/uslugi/GPO/23.png}{Konfiguracja dysku dla IT – cz. 1}{GPO23}
\fg{rys/uslugi/GPO/24.png}{Konfiguracja dysku dla IT – cz. 2}{GPO24}
\clearpage
\fg{rys/uslugi/GPO/25.png}{Konfiguracja dysku dla IT – cz. 3}{GPO25}

\subsubsection{Zmienna środowiskowa i tworzenie folderu}
\fg{rys/uslugi/GPO/26.png}{Utworzenie nowej polityki}{GPO26}
\clearpage
\fg{rys/uslugi/GPO/27.png}{Nazwa polityki}{GPO27}
\fg{rys/uslugi/GPO/28.png}{Edycja polityki}{GPO28}
\clearpage

\fg{rys/uslugi/GPO/29.png}{Wybranie User Configuration $\rightarrow$ Preferences $\rightarrow$ Windows Settings $\rightarrow$ Environment $\rightarrow$ New $\rightarrow$ Environment Variable}{GPO29}
\fg{rys/uslugi/GPO/30.png}{Konfiguracja zmiennej środowiskowej}{GPO30}
\clearpage
\fg{rys/uslugi/GPO/31.png}{Test zmiennej środowiskowej}{GPO31}
\fg{rys/uslugi/GPO/32.png}{Utworzenie nowej polityki}{GPO26}
\clearpage
\fg{rys/uslugi/GPO/33.png}{Nazwa polityki}{GPO27}
\fg{rys/uslugi/GPO/34.png}{Edycja polityki}{GPO28}
\clearpage
\fg{rys/uslugi/GPO/35.png}{Wybranie User Configuration $\rightarrow$ Preferences $\rightarrow$ Windows Settings $\rightarrow$ Folders $\rightarrow$ New $\rightarrow$ Folder}{GPO29}
\fg{rys/uslugi/GPO/36.png}{Konfiguracja folderu}{GPO30}
\clearpage
\fg{rys/uslugi/GPO/37.png}{Stan przed zastosowaniem polityki}{GPO31}
\fg{rys/uslugi/GPO/38.png}{Zastosowanie polityki}{GPO31}
\clearpage
\fg{rys/uslugi/GPO/39.png}{Stan po zastosowaniu polityki}{GPO31}


\subsection{Instalacja usługi Dynamic Host Configuration Protocol (DHCP)}
Na klastrze zainstalowano usługę Dynamic Host Configuration Protocol (DHCP) w celu automatycznego przydzielania adresów IP oraz konfiguracji sieci dla urządzeń w sieci lokalnej. 
\fg{rys/uslugi/DHCP/1.png}{Otwarcie kreatora instalacji roli na klastrze}{DHCP1}
\fg{rys/uslugi/DHCP/2.png}{Pierwsza strona kreatora}{DHCP2}
\clearpage
\fg{rys/uslugi/DHCP/3.png}{Instalacja oparta na rolach lub funkcjach}{DHCP3}
\fg{rys/uslugi/DHCP/4.png}{Wybór serwerów}{DHCP4}
\clearpage
\fg{rys/uslugi/DHCP/5.png}{Wybranie funkcji DHCP}{DHCP5}
\fg{rys/uslugi/DHCP/6.png}{Ekran dotyczący DHCP Server}{DHCP6}
\clearpage
\fg{rys/uslugi/DHCP/7.png}{Rozpoczęcie instalacji}{DHCP7}
\fg{rys/uslugi/DHCP/8.png}{Powrót do Failover Cluster Manager i kreatora funkcji DHCP – nadanie nazwy serwera i ustalenie adresu IP}{DHCP8}
\clearpage
\fg{rys/uslugi/DHCP/9.png}{Dodanie dysku 5GB do maszyny wirtualnej}{DHCP9}
\fg{rys/uslugi/DHCP/10.png}{Widok dysku w ustawieniach maszyny wirtualnej}{DHCP10}
\clearpage
\fg{rys/uslugi/DHCP/11.png}{Initializacja dysku}{DHCP11}
\fg{rys/uslugi/DHCP/12.png}{Formatowanie dysku i nadanie etykiety DHCP}{DHCP12}
\clearpage
\fg{rys/uslugi/DHCP/13.png}{Przejście do New iSCSI Virtual Disk}{DHCP13}
\fg{rys/uslugi/DHCP/14.png}{Dokończenie konfiguracji roli DHCP na SN1 i SN2}{DHCP14}
\clearpage
\fg{rys/uslugi/DHCP/15.png}{Wybór dysku do konfiguracji $\rightarrow$ E: w New iSCSI Virtual Disk}{DHCP15}
\fg{rys/uslugi/DHCP/16.png}{Wybór nazwy dysku $\rightarrow$ DHCP}{DHCP16}
\clearpage
\fg{rys/uslugi/DHCP/17.png}{Wybór rozmiaru dysku $\rightarrow$ 5GB}{DHCP17}
\fg{rys/uslugi/DHCP/18.png}{Nowy cel}{DHCP18}
\clearpage
\fg{rys/uslugi/DHCP/19.png}{Nazwa celu DHCP-SN1-SN2-KLASTER}{DHCP19}
\fg{rys/uslugi/DHCP/20.png}{Dodanie serwerów dostępu SN1 i SN2}{DHCP20}
\clearpage
\fg{rys/uslugi/DHCP/21.png}{Ekran potwierdzenia ustawień iSCSI}{DHCP21}
\fg{rys/uslugi/DHCP/22.png}{Połaćzenie z dyskiem na SN1 i SN2}{DHCP22}
\clearpage
\fg{rys/uslugi/DHCP/23.png}{Automatyczna konfiguracja dysku na SN1 i SN2}{DHCP23}
\fg{rys/uslugi/DHCP/24.png}{Przestawienie dysków na online na SN1}{DHCP24}
\clearpage
\fg{rys/uslugi/DHCP/25.png}{Inicjalizacja dysku na SN1}{DHCP25}
\fg{rys/uslugi/DHCP/26.png}{Formatowanie i nadanie etykiety dysku na SN1}{DHCP26}
\clearpage
\fg{rys/uslugi/DHCP/27.png}{Dodanie dysku do klastra}{DHCP27}
\fg{rys/uslugi/DHCP/28.png}{Wynik dodania dysku do klastra}{DHCP28}
\clearpage
\fg{rys/uslugi/DHCP/29.png}{Kontynuacja dodawania roli DHCP dodanie dysku klastra do roli}{DHCP29}
\fg{rys/uslugi/DHCP/30.png}{Ekran potwierdzenia}{DHCP30}
\clearpage
\fg{rys/uslugi/DHCP/31.png}{Podsumowanie dodania roli DHCP}{DHCP31}
\fg{rys/uslugi/DHCP/32.png}{Otworzenie zarządzania DHCP}{DHCP32}
\clearpage
\fg{rys/uslugi/DHCP/33.png}{Dodanie nowego zakresu}{DHCP33}
\fg{rys/uslugi/DHCP/34.png}{Konfiguracja zakresu – nazwa}{DHCP34}
\clearpage
\fg{rys/uslugi/DHCP/35.png}{Konfiguracja zakresu – zakres adresów}{DHCP35}
\fg{rys/uslugi/DHCP/36.png}{Konfiguracja wykluczeń}{DHCP36}
\clearpage
\fg{rys/uslugi/DHCP/37.png}{Konfiguracja zakresu – długość dzierżawy}{DHCP37}
\fg{rys/uslugi/DHCP/38.png}{Potwierdzenie konfiguracji opcji DHCP}{DHCP38}
\clearpage
\fg{rys/uslugi/DHCP/39.png}{Konfiguracja zakresu – brama domyślna}{DHCP39}
\fg{rys/uslugi/DHCP/40.png}{Konfiguracja zakresu – serwer DNS}{DHCP40}
\clearpage
\fg{rys/uslugi/DHCP/41.png}{Konfiguracja zakresu – serwer WINS}{DHCP41}
\fg{rys/uslugi/DHCP/42.png}{Aktywacja zakresu}{DHCP42}
\clearpage
\fg{rys/uslugi/DHCP/43.png}{Sprawdzenie działania DHCP – cz. 1}{DHCP43}
\fg{rys/uslugi/DHCP/44.png}{Sprawdzenie działania DHCP – cz. 2}{DHCP44}


\subsection{Instalacja usługi Windows Deployment Services (WDS)}
\fg{rys/uslugi/WDS/1.png}{Instalacja oparta na rolach lub funkcjach}{WDS1}
\fg{rys/uslugi/WDS/2.png}{Wybór serwera SDC06}{WDS2}
\clearpage
\fg{rys/uslugi/WDS/3.png}{Wybranie funkcji Windows Deployment Services}{WDS3}
\fg{rys/uslugi/WDS/4.png}{Role Services – wybór obu}{WDS4}
\clearpage
\fg{rys/uslugi/WDS/5.png}{Rozpoczęcie instalacji}{WDS5}
\fg{rys/uslugi/WDS/6.png}{Instalacja zakończona sukcesem}{WDS6}
\clearpage
\fg{rys/uslugi/WDS/7.png}{Otwarcie Windows Deployment Services Console}{WDS7}
\fg{rys/uslugi/WDS/8.png}{Konfiguracja serwera}{WDS8}
\clearpage
\fg{rys/uslugi/WDS/9.png}{Zintegrowanie WDS z Active Directory}{WDS9}
\fg{rys/uslugi/WDS/10.png}{Wybór ścieżki do magazynu obrazów}{WDS10}
\clearpage
\fg{rys/uslugi/WDS/11.png}{Odpowaadanie wszystkim hostom}{WDS11}
\fg{rys/uslugi/WDS/12.png}{Dodanie obrazu instalacyjnego}{WDS12}
\clearpage
\fg{rys/uslugi/WDS/13.png}{Utworzenie katalogu ISO na dysku C\: i przerzucenie pliku ISO do maszyny wirtualnej}{WDS13}
\fg{rys/uslugi/WDS/14.png}{Dodanie obrazu instalacyjnego}{WDS14}
\clearpage
\fg{rys/uslugi/WDS/15.png}{Stworzenie nowej grupy}{WDS15}
\fg{rys/uslugi/WDS/16.png}{Zamontoawnie obrazu}{WDS16}
\clearpage
\fg{rys/uslugi/WDS/21.png}{Skopiowanie plików z obrazu do katalogu C:\textbackslash sources}{WDS17}
\fg{rys/uslugi/WDS/18.png}{Wybór obrazu instalacyjnego}{WDS18}
\clearpage
\fg{rys/uslugi/WDS/19.png}{Przeklikanie kreatora do końca}{WDS19}
\fg{rys/uslugi/WDS/20.png}{Dodanie obrazu bootowalnego}{WDS20}
\clearpage
\fg{rys/uslugi/WDS/21.png}{Wybór obrazu boot.wim}{WDS21}
\fg{rys/uslugi/WDS/22.png}{Przeklikanie kreatora do końca}{WDS22}
\clearpage
\fg{rys/uslugi/WDS/23.png}{Stworzenie nowej maszyny wirtualnej}{WDS23}
\fg{rys/uslugi/WDS/24.png}{Stworzenie nowej maszyny wirtualnej – konfiguracja}{WDS24}
\clearpage
\fg{rys/uslugi/WDS/25.png}{Stworzenie nowej maszyny wirtualnej – wybór wewnętrznej sieci}{WDS25}
\fg{rys/uslugi/WDS/26.png}{Uruchomienie maszyny wirtualnej}{WDS26}
\clearpage
\fg{rys/uslugi/WDS/27.png}{Uruchomienie maszyny wirtualnej – załadowanie obrazu przez sieć}{WDS27}



\subsection{Instalacja serwera wydruku}

\subsection{Instalacja serwera WWW}

\subsection{Instalacja WordPress}


	\newpage
\section{Testy działania wdrożonych usług}	%5
% (W podpunktach zamieścić zrzuty ekranów pokazujące działanie wdrożonych usług)

\subsection{AD DS (Active Directory Domain Services)}
\fg{rys/test/DNS/1.png}{Udało się dołączyć do domeny}{DNS01}
\clearpage

\subsection{DNS (Domain Name System)}
\fg{rys/test/DNS/1.png}{Komenda nslookup na komputerze uzytkownika}{DNS01}
\clearpage

\subsection{DHCP (Dynamic Host Configuration Protocol)}
\fg{rys/test/DHCP/1.png}{Adres został uzyskany poprzez DHCP}{DHCP01}
\clearpage

\subsection{GPO (Group Policy Objects)}
\fg{rys/test/GPO/1.png}{Została dodana zmienna FIRMA}{GPO01}
\clearpage

\subsection{WDS (Windows Deployment Services)}
\fg{rys/test/WDS/1.png}{Można zainstalować system Windows z obrazu zamieszczonego na serwerze}{WDS01}
\clearpage

\subsection{RAID 1 (Redundant Array of Independent Disks)}
\fg{rys/test/RAID1/1.png}{Widok zdublowanych macierzy w zarządzaniu dyskami}{RAID501}
\clearpage

\subsection{iSCSI (Internet Small Computer System Interface)}
\fg{rys/test/iSCSI/1.png}{Udostępniony przez iSCSI dysk jest widoczny na komputerze użytkownika}{iSCSI01}
\clearpage

\subsection{Serwer Wydruków (Print Server)}
\fg{rys/test/PrintServer/1.png}{Dodana drukarka na komputerze użytkownika}{PrintServer01}
\clearpage

\subsection{Serwer WWW (Web Server)}
\fg{rys/test/IIS/1.png}{Strona jest dostępna z komputera klienta}{WebServer01}
\clearpage

\subsection{FAILOVER CLUSTERING z rolami: SFS, DHCP Server}
\fg{rys/test/FailoverClustering/1.png}{SFS działa to oznacza że klaster działa}{FailoverClustering01}
\clearpage

\subsection{Automatyzacja - Skrypty PowerShell}
\fg{rys/test/Skrypt/1.png}{Stan przed wykonaniem skryptu}{PowerShell01}
\fg{rys/test/Skrypt/2.png}{Stan po wykonaniu skryptu}{PowerShell01}
\clearpage

\subsection{Wordpress}
\fg{rys/test/WP/1.png}{Strona główna Wordpressa}{Wordpress01}
\clearpage
\newpage
	\newpage
\section{Wnioski}	%5
%Npisać wnioski końcowe z przeprowadzonego projektu, 





%%%%%%%%%%%%%%%%%%% koniec treść główna dokumentu %%%%%%%%%%%%%%%%%%%%%
\newpage
% \addcontentsline{toc}{section}{Literatura}
% Modified by: Maciej Wójs  
\printbibliography[heading=bibnumbered, label=Literatura, title=Literatura]

\newpage
\hypersetup{linkcolor=black}
\renewcommand{\cftparskip}{3pt}
\clearpage
\renewcommand{\cftloftitlefont}{\Large\bfseries\sffamily}
\listoffigures
\addcontentsline{toc}{section}{Spis rysunków}
\thispagestyle{fancy}

\newpage
\renewcommand{\cftlottitlefont}{\Large\bfseries\sffamily}
\def\listtablename{Spis tabel}
\addcontentsline{toc}{section}{Spis tabel}\listoftables
\thispagestyle{fancy}

\newpage
\renewcommand{\cftlottitlefont}{\Large\bfseries\sffamily}
\renewcommand\lstlistlistingname{Spis listingów}
\addcontentsline{toc}{section}{Spis listingów}\lstlistoflistings
\thispagestyle{fancy}

%lista rzeczy do zrobienia: wypisuje na koñcu dokumentu, patrz: pakiet todo.sty
\todos
%koniec listy rzeczy do zrobienia
\end{document}
